\documentclass[10pt]{article} 
\usepackage{array,natbib}
\usepackage{amsmath, amsthm, amssymb}

\usepackage[pdftex,colorlinks]{hyperref}

\begin{document}

\title{DSGE Models with Dynare++. A Tutorial.}

\author{Ondra Kamen\'\i k}

\date{January 2011}
\maketitle

\tableofcontents

\section{Setup}

The Dynare++ setup procedure is pretty straightforward as Dynare++ is included in the Dynare installation 
packages which can be downloaded from \url{http://www.dynare.org}. Take the following steps:
\begin{enumerate}
\item Add the {\tt dynare++} subdirectory of the root Dynare installation directory to the your
operating system path. This ensures that your OS will find the {\tt dynare++} executable.
\item If you have MATLAB and want to run custom simulations (see \ref{custom}),
  then you need to add to your MATLAB path the {\tt dynare++} subdirectory of
  the root Dynare installation directory, and also directory containing the
  \texttt{dynare\_simul\_} MEX file (note the trailing underscore). The easiest
  way to add the latter is to run Dynare once in your MATLAB session (even
  without giving it any MOD file).
\end{enumerate}

\section{Sample Session}

As an example, let us take a simple DSGE model whose dynamic
equilibrium is described by the following first order conditions:

\begin{align*}
&c_t\theta h_t^{1+\psi} = (1-\alpha)y_t\cr
&\beta E_t\left[\frac{\exp(b_t)c_t}{\exp(b_{t+1})c_{t+1}}
\left(\exp(b_{t+1})\alpha\frac{y_{t+1}}{k_{t+1}}+1-\delta\right)\right]=1\cr
&y_t=\exp(a_t)k_t^\alpha h_t^{1-\alpha}\cr
&k_{t}=\exp(b_{t-1})(y_{t-1}-c_{t-1})+(1-\delta)k_{t-1}\cr
&a_t=\rho a_{t-1}+\tau b_{t-1}+\epsilon_t\cr
&b_t=\tau a_{t-1}+\rho b_{t-1}+\nu_t
\end{align*}

\label{timing}
The timing of this model is that the exogenous shocks $\epsilon_t$,
and $\nu_t$ are observed by agents in the beginning of period $t$ and
before the end of period $t$ all endogenous variables with index $t$
are decided. The expectation operator $E_t$ works over the information
accumulated just before the end of the period $t$ (this includes
$\epsilon_t$, $\nu_t$ and all endogenous variables with index $t$).

The exogenous shocks $\epsilon_t$ and $\nu_t$ are supposed to be
serially uncorrelated with zero means and time-invariant
variance-covariance matrix. In Dynare++, these variables are called
exogenous; all other variables are endogenous. Now we are prepared to
start writing a model file for Dynare++, which is an ordinary text
file and could be created with any text editor.

The model file starts with a preamble declaring endogenous and
exogenous variables, parameters, and setting values of the
parameters. Note that one can put expression on right hand sides. The
preamble follows:

{\small
\begin{verbatim}
var Y, C, K, A, H, B;
varexo EPS, NU;

parameters beta, rho, beta, alpha, delta, theta, psi, tau;
alpha = 0.36;
rho   = 0.95;
tau   = 0.025;
beta  = 1/(1.03^0.25);
delta = 0.025;
psi   = 0;
theta = 2.95;
\end{verbatim}
}

The section setting values of the parameters is terminated by a
beginning of the {\tt model} section, which states all the dynamic
equations. A timing convention of a Dynare++ model is the same as the
timing of our example model, so we may proceed with writing the model
equations. The time indexes of $c_{t-1}$, $c_t$, and $c_{t+1}$ are
written as {\tt C(-1)}, {\tt C}, and {\tt C(1)} resp. The {\tt model}
section looks as follows:

{\small
\begin{verbatim}
model;
C*theta*H^(1+psi) = (1-alpha)*Y;
beta*exp(B)*C/exp(B(1))/C(1)*
  (exp(B(1))*alpha*Y(1)/K(1)+1-delta) = 1;
Y = exp(A)*K^alpha*H^(1-alpha);
K = exp(B(-1))*(Y(-1)-C(-1)) + (1-delta)*K(-1);
A = rho*A(-1) + tau*B(-1) + EPS;
B = tau*A(-1) + rho*B(-1) + NU;
end;
\end{verbatim}
}

At this point, almost all information that Dynare++ needs has been
provided. Only three things remain to be specified: initial values of
endogenous variables for non-linear solver, variance-covariance matrix
of the exogenous shocks and order of the Taylor approximation. Since
the model is very simple, there is a closed form solution for the
deterministic steady state. We use it as initial values for the
non-linear solver. Note that the expressions on the right hand-sides in
{\tt initval} section can reference values previously calculated. The
remaining portion of the model file looks as follows:

{\small
\begin{verbatim}
initval;
A = 0;
B = 0;
H = ((1-alpha)/(theta*(1-(delta*alpha)
     /(1/beta-1+delta))))^(1/(1+psi));
Y = (alpha/(1/beta-1+delta))^(alpha/(1-alpha))*H;
K = alpha/(1/beta-1+delta)*Y;
C = Y - delta*K;
end;

vcov = [
  0.0002  0.00005;
  0.00005 0.0001
];

order = 7;
\end{verbatim}
}

Note that the order of rows/columns of the variance-covariance matrix
corresponds to the ordering of exogenous variables in the {\tt varexo}
declaration. Since the {\tt EPS} was declared first, its variance is
$0.0002$, and the variance of {\tt NU} is $0.0001$.

Let the model file be saved as {\tt example1.mod}. Now we are prepared
to solve the model. At the operating system command
prompt\footnote{Under Windows it is a {\tt cmd} program, under Unix it
is any shell} we issue a command:

{\small
\begin{verbatim}
dynare++ example1.mod
\end{verbatim}
}

When the program is finished, it produces two output files: a journal
file {\tt example1.jnl} and a Matlab MAT-4 {\tt example1.mat}. The
journal file contains information about time, memory and processor
resources needed for all steps of solution. The output file is more
interesting. It contains various simulation results. It can be loaded
into Matlab or Scilab and examined.%
\footnote{For Matlab {\tt load example1.mat}, for Scilab {\tt
mtlb\_load example1.mat}} The following examples are done in Matlab,
everything would be very similar in Scilab.

Let us first examine the contents of the MAT file:
{\small
\begin{verbatim}
>> load example1.mat
>> who

Your variables are:

dyn_g_1            dyn_i_Y            dyn_npred          
dyn_g_2            dyn_irfm_EPS_mean  dyn_nstat          
dyn_g_3            dyn_irfm_EPS_var   dyn_shocks         
dyn_g_4            dyn_irfm_NU_mean   dyn_ss             
dyn_g_5            dyn_irfm_NU_var    dyn_state_vars     
dyn_i_A            dyn_irfp_EPS_mean  dyn_steady_states  
dyn_i_B            dyn_irfp_EPS_var   dyn_vars           
dyn_i_C            dyn_irfp_NU_mean   dyn_vcov           
dyn_i_EPS          dyn_irfp_NU_var    dyn_vcov_exo       
dyn_i_H            dyn_mean           
dyn_i_K            dyn_nboth          
dyn_i_NU           dyn_nforw          
\end{verbatim}
}

All the variables coming from one MAT file have a common prefix. In
this case it is {\tt dyn}, which is Dynare++ default. The prefix can
be changed, so that the multiple results could be loaded into one Matlab
session.

In the default setup, Dynare++ solves the Taylor approximation to the
decision rule and calculates unconditional mean and covariance of the
endogenous variables, and generates impulse response functions. The
mean and covariance are stored in {\tt dyn\_mean} and {\tt
dyn\_vcov}. The ordering of the endogenous variables is given by {\tt
dyn\_vars}.

In our example, the ordering is

{\small
\begin{verbatim}
>> dyn_vars
dyn_vars =
H
A
Y
C
K
B
\end{verbatim}
}

and unconditional mean and covariance are

{\small
\begin{verbatim}
>> dyn_mean
dyn_mean =
    0.2924
    0.0019
    1.0930
    0.8095
   11.2549
    0.0011
>> dyn_vcov
dyn_vcov =
    0.0003    0.0006    0.0016    0.0004    0.0060    0.0004
    0.0006    0.0024    0.0059    0.0026    0.0504    0.0012
    0.0016    0.0059    0.0155    0.0069    0.1438    0.0037
    0.0004    0.0026    0.0069    0.0040    0.0896    0.0016
    0.0060    0.0504    0.1438    0.0896    2.1209    0.0405
    0.0004    0.0012    0.0037    0.0016    0.0405    0.0014
\end{verbatim}
}

The ordering of the variables is also given by indexes starting with
{\tt dyn\_i\_}. Thus the mean of capital can be retrieved as

{\small
\begin{verbatim}
>> dyn_mean(dyn_i_K)
ans =
   11.2549
\end{verbatim}
}

\noindent and covariance of labor and capital by

{\small
\begin{verbatim}
>> dyn_vcov(dyn_i_K,dyn_i_H)
ans =
    0.0060
\end{verbatim}
}

The impulse response functions are stored in matrices as follows
\begin{center}
\begin{tabular}{|l|l|}
\hline
matrix& response to\\
\hline
{\tt dyn\_irfp\_EPS\_mean}& positive impulse to {\tt EPS}\\
{\tt dyn\_irfm\_EPS\_mean}& negative impulse to {\tt EPS}\\
{\tt dyn\_irfp\_NU\_mean}& positive impulse to {\tt NU}\\
{\tt dyn\_irfm\_NU\_mean}& negative impulse to {\tt NU}\\
\hline
\end{tabular}
\end{center}
All shocks sizes are one standard error. Rows of the matrices
correspond to endogenous variables, columns correspond to
periods. Thus capital response to a positive shock to {\tt EPS} can be
plotted as

{\small
\begin{verbatim}
plot(dyn_irfp_EPS_mean(dyn_i_K,:));
\end{verbatim}
}

The data is in units of the respective variables, so in order to plot
the capital response in percentage changes from the decision rule's
fix point (which is a vector {\tt dyn\_ss}), one has to issue the
commands:

{\small
\begin{verbatim}
Kss=dyn_ss(dyn_i_K);
plot(100*dyn_irfp_EPS_mean(dyn_i_K,:)/Kss);
\end{verbatim}
}

The plotted impulse response shows that the model is pretty persistent
and that the Dynare++ default for a number of simulated periods is not
sufficient. In addition, the model persistence puts in doubt also a
number of simulations. The Dynare++ defaults can be changed when
calling Dynare++, in operating system's command prompt, we issue a
command:

{\small
\begin{verbatim}
dynare++ --per 300 --sim 150 example1.mod
\end{verbatim}
}

\noindent This sets the number of simulations to $150$ and the number
of periods to $300$ for each simulation giving $45000$ total simulated
periods.

\section{Sample Optimal Policy Session}
\label{optim_tut}

Suppose that one wants to solve the following optimal policy problem
with timeless perspective.\footnote{See \ref{ramsey} on how to solve
Ramsey optimality problem within this framework} The following
optimization problem is how to choose capital taxes financing public
good to maximize agent's utility from consumption good and public
good. The problem takes the form:
\begin{align*}
\max_{\{\tau_t\}_{t_0}^\infty} 
E_{t_0}\sum_{t=t_0}^\infty &\beta^{t-t_0}\left(u(c_t)+av(g_t)\right)\\
\hbox{subject\ to}&\\
u'(c_t) &=
\beta E_t\left[u'(c_{t+1})\left(1-\delta+f'(k_{t+1})(1-\alpha\tau_{t+1})\right)\right]\\
K_t &= (1-\delta)K_{t-1} + (f(K_{t-1}) - c_{t-1} - g_{t-1})\\
g_t &= \tau_t\alpha f(K_t),\\
\hbox{where\ } t & = \ldots,t_0-1,t_0,t_0+1,\ldots
\end{align*}
$u(c_t)$ is utility from consuming the consumption good, $v(g_t)$ is
utility from consuming the public good, $f(K_t)$ is a production
function $f(K_t) = Z_tK_t^\alpha$. $Z_t$ is a technology shock modeled
as AR(1) process. The three constraints come from the first order
conditions of a representative agent. We suppose that it pursues a
different objective, namely lifetime utility involving only
consumption $c_t$. The representative agents chooses between
consumption and investment. It rents the capital to firms and supplies
constant amount of labour. All output is paid back to consumer in form
of wage and capital rent. Only the latter is taxed. We suppose that
the optimal choice has been taking place from infinite past and will
be taking place for ever. Further we suppose the same about the
constraints.

Let us choose the following functional forms:
\begin{eqnarray*}
u(c_t) &=& \frac{c_t^{1-\eta}}{1-\eta}\\
v(g_t) &=& \frac{g_t^{1-\phi}}{1-\phi}\\
f(K_t) &=& K_t^\alpha
\end{eqnarray*}

Then the problem can be coded into Dynare++ as follows. We start with
a preamble which states all the variables, shocks and parameters:
{\small
\begin{verbatim}
var C G K TAU Z;

varexo EPS;

parameters eta beta alpha delta phi a rho; 

eta = 2;
beta = 0.99;
alpha = 0.3;
delta = 0.10;
phi = 2.5;
a = 0.1;
rho = 0.7;
\end{verbatim}
}

Then we specify the planner's objective and the discount factor in the
objective. The objective is an expression (possibly including also
variable leads and lags), and the discount factor must be one single
declared parameter:
{\small
\begin{verbatim}
planner_objective C^(1-eta)/(1-eta) + a*G^(1-phi)/(1-phi);

planner_discount beta;
\end{verbatim}
}

The model section will contain only the constraints of the social
planner. These are capital accumulation, identity for the public
product, AR(1) process for $Z_t$ and the first order condition of the
representative agent (with different objective).
{\small
\begin{verbatim}
model;
K = (1-delta)*K(-1) + (exp(Z(-1))*K(-1)^alpha - C(-1) - G(-1));
G = TAU*alpha*K^alpha;
Z = rho*Z(-1) + EPS;
C^(-eta) = beta*C(+1)^(-eta)*(1-delta +
           exp(Z(+1))*alpha*K(+1)^(alpha-1)*(1-alpha*TAU(+1)));
end;
\end{verbatim}
}

Now we have to provide a good guess for non-linear solver calculating
the deterministic steady state. The model's steady state has a closed
form solution if the taxes are known. So we provide a guess for
taxation {\tt TAU} and then use the closed form solution for capital,
public good and consumption:\footnote{Initial guess for Lagrange
multipliers and some auxiliary variables is calculated automatically. See
\ref{opt_init} for more details.}
{\small
\begin{verbatim}
initval;
TAU = 0.70;
K = ((delta+1/beta-1)/(alpha*(1-alpha*TAU)))^(1/(alpha-1));
G = TAU*alpha*K^alpha;
C =  K^alpha - delta*K - G;
Z = 0;
\end{verbatim}
}

Finally, we have to provide the order of approximation, and the
variance-covariance matrix of the shocks (in our case we have only one
shock):
{\small
\begin{verbatim}
order = 4;

vcov = [
	0.01
];
\end{verbatim}
}

After this model file has been run, we can load the resulting MAT-file
into the Matlab (or Scilab) and examine its contents:
{\small
\begin{verbatim}
>> load kp1980_2.mat
>> who

Your variables are:

dyn_g_1            dyn_i_MULT1        dyn_nforw          
dyn_g_2            dyn_i_MULT2        dyn_npred          
dyn_g_3            dyn_i_MULT3        dyn_nstat          
dyn_g_4            dyn_i_TAU          dyn_shocks         
dyn_i_AUX_3_0_1    dyn_i_Z            dyn_ss             
dyn_i_AUX_4_0_1    dyn_irfm_EPS_mean  dyn_state_vars     
dyn_i_C            dyn_irfm_EPS_var   dyn_steady_states  
dyn_i_EPS          dyn_irfp_EPS_mean  dyn_vars           
dyn_i_G            dyn_irfp_EPS_var   dyn_vcov           
dyn_i_K            dyn_mean           dyn_vcov_exo       
dyn_i_MULT0        dyn_nboth          
\end{verbatim}
}

The data dumped into the MAT-file have the same structure as in the
previous example of this tutorial. The only difference is that
Dynare++ added a few more variables. Indeed:
{\small
\begin{verbatim}
>> dyn_vars
dyn_vars =
MULT1    
G        
MULT3    
C        
K        
Z        
TAU      
AUX_3_0_1
AUX_4_0_1
MULT0    
MULT2    
\end{verbatim}
}
Besides the five variables declared in the model ({\tt C}, {\tt G},
{\tt K}, {\tt TAU}, and {\tt Z}), Dy\-na\-re++ added 6 more, four as Lagrange
multipliers of the four constraints, two as auxiliary variables for
shifting in time. See \ref{aux_var} for more details.

The structure and the logic of the MAT-file is the same as these new 6
variables were declared in the model file and the file is examined in
the same way.

For instance, let us examine the Lagrange multiplier of the optimal
policy associated with the consumption first order condition. Recall
that the consumers' objective is different from the policy
objective. Therefore, the constraint will be binding and the
multiplier will be non-zero. Indeed, its deterministic steady state,
fix point and mean are as follows:
{\small
\begin{verbatim}
>> dyn_steady_states(dyn_i_MULT3,1)
ans =
   -1.3400
>> dyn_ss(dyn_i_MULT3)
ans =
   -1.3035
>> dyn_mean(dyn_i_MULT3)
ans =
   -1.3422
\end{verbatim}
}

\section{What Dynare++ Calculates}
\label{dynpp_calc}

Dynare++ solves first order conditions of a DSGE model in the recursive form:
\begin{equation}\label{focs}
E_t[f(y^{**}_{t+1},y_t,y^*_{t-1},u_t)]=0,
\end{equation}
where $y$ is a vector of endogenous variables, and $u$ a vector of
exogenous variables. Some of elements of $y$ can occur at time $t+1$,
these are $y^{**}$. Elements of $y$ occurring at time $t-1$ are denoted
$y^*$. The exogenous shocks are supposed to be serially independent
and normally distributed $u_t\sim N(0,\Sigma)$.

The solution of this dynamic system is a decision rule
\[
y_t=g(y^*_{t-1},u_t)
\]
Dynare++ calculates a Taylor approximation of this decision rule of a
given order. The approximation takes into account deterministic
effects of future volatility, so a point about which the Taylor
approximation is done will be different from the fix point $y$ of the rule
yielding $y=g(y^*,0)$.

The fix point of a rule corresponding to a model with $\Sigma=0$ is
called {\it deterministic steady state} denoted as $\bar y$. In
contrast to deterministic steady state, there is no consensus in
literature how to call a fix point of the rule corresponding to a
model with non-zero $\Sigma$. I am tempted to call it {\it stochastic
  steady state}, however, it might be confused with unconditional mean
or with steady distribution. So I will use a term {\it fix point} to
avoid a confusion.

By default, Dynare++ solves the Taylor approximation about the
deterministic steady state. Alternatively, Dynare++ can split the
uncertainty to a few steps and take smaller steps when calculating the
fix points. This is controlled by an option {\tt --steps}. For the
brief description of the second method, see \ref{multistep_alg}.

\subsection{Decision Rule Form}
\label{dr_form}

In case of default solution algorithm (approximation about the
deterministic steady state $\bar y$), Dynare++ calculates the higher
order derivatives of the equilibrium rule to get a decision rule of
the following form. In Einstein notation, it is:
\[
y_t-\bar y = \sum_{i=0}^k\frac{1}{i!}\left[g_{(y^*u)^i}\right]
_{\alpha_1\ldots\alpha_i}
\prod_{j=1}^i\left[\begin{array}{c} y^*_{t-1}-\bar y^*\\ u_t \end{array}\right]
^{\alpha_j}
\]

Note that the ergodic mean will be different from the deterministic
steady state $\bar y$ and thus deviations $y^*_{t-1}-\bar y^*$ will
not be zero in average. This implies that in average we will commit
larger round off errors than if we used the decision rule expressed in
deviations from a point closer to the ergodic mean. Therefore, by
default, Dynare++ recalculates this rule and expresses it in
deviations from the stochastic fix point $y$.
\[
y_t-y = \sum_{i=1}^k\frac{1}{i!}\left[\tilde g_{(y^*u)^i}\right]
_{\alpha_1\ldots\alpha_i}
\prod_{j=1}^i\left[\begin{array}{c} y^*_{t-1}-y^*\\ u_t \end{array}\right]
^{\alpha_j}
\]
Note that since the rule is centralized around its fix point, the
first term (for $i=0$) drops out.

Also note, that this rule mathematically equivalent to the rule
expressed in deviations from the deterministic steady state, and still
it is an approximation about the deterministic steady state. The fact
that it is expressed in deviations from a different point should not
be confused with the algorithm in \ref{multistep_alg}.

This centralization can be avoided by invoking {\tt --no-centralize}
command line option.

\subsection{Taking Steps in Volatility Dimension}
\label{multistep_alg}

For models, where volatility of the exogenous shocks plays a big
role, the approximation about deterministic steady state can be poor,
since the equilibrium dynamics can be very different from the dynamics
in the vicinity of the perfect foresight (deterministic steady state).

Therefore, Dynare++ has on option {\tt --steps} triggering a multistep
algorithm. The algorithm splits the volatility to a given number of
steps. Dynare++ attempts to calculate approximations about fix points
corresponding to these levels of volatility. The problem is that if we
want to calculate higher order approximations about fix points
corresponding to volatilities different from zero (as in the case of
deterministic steady state), then the derivatives of lower orders
depend on derivatives of higher orders with respect to forward looking
variables. The multistep algorithm in each step approximates the
missing higher order derivatives with extrapolations based on the
previous step.

In this way, the approximation of the stochastic fix point and the
derivatives about this fix point are obtained. It is difficult to a
priori decide whether this algorithm yields a better decision
rule. Nothing is guaranteed, and the resulted decision rule should be
checked with a numerical integration. See \ref{checks}.

\subsection{Simulating the Decision Rule}

After some form of a decision rule is calculated, it is simulated to
obtain draws from ergodic (unconditional) distribution of endogenous
variables. The mean and the covariance are reported. There are two
ways how to calculate the mean and the covariance. The first one is to
store all simulated samples and calculate the sample mean and
covariance. The second one is to calculate mean and the covariance in
the real-time not storing the simulated sample. The latter case is
described below (see \ref{rt_simul}).

The stored simulated samples are then used for impulse response
function calculations. For each shock, the realized shocks in these
simulated samples (control simulations) are taken and an impulse is
added and the new realization of shocks is simulated. Then the control
simulation is subtracted from the simulation with the impulse. This is
done for all control simulations and the results are averaged. As the
result, we get an expectation of difference between paths with impulse
and without impulse. In addition, the sample variances are
reported. They might be useful for confidence interval calculations.

For each shock, Dynare++ calculates IRF for two impulses, positive and
negative. Size of an impulse is one standard error of a respective
shock.

The rest of this subsection is divided to three parts giving account
on real-time simulations, conditional simulations, and on the way how
random numbers are generated resp.

\subsubsection{Simulations With Real-Time Statistics}
\label{rt_simul}

When one needs to simulate large samples to get a good estimate of
unconditional mean, simulating the decision rule with statistics
calculated in real-time comes handy. The main reason is that the
storing of all simulated samples may not fit into the available
memory.

The real-time statistics proceed as follows: We model the ergodic
distribution as having normal distribution $y\sim N(\mu,\Sigma)$. Further,
the parameters $\mu$ and $\Sigma$ are modelled as:
\begin{eqnarray*}
  \Sigma &\sim& {\rm InvWishart}_\nu(\Lambda)\\
  \mu|\Sigma &\sim& N(\bar\mu,\Sigma/\kappa) \\ 
\end{eqnarray*}
This model of $p(\mu,\Sigma)$ has an advantage of conjugacy, i.e. a
prior distribution has the same form as posterior. This property is
used in the calculation of real-time estimates of $\mu$ and $\Sigma$,
since it suffices to maintain only the parameters of $p(\mu,\Sigma)$
conditional observed draws so far. The parameters are: $\nu$,
$\Lambda$, $\kappa$, and $\bar\mu$.

The mean of $\mu,\Sigma|Y$, where $Y$ are all the draws (simulated
periods) is reported.

\subsubsection{Conditional Distributions}
\label{cond_dist}

Starting with version 1.3.6, Dynare++ calculates variable
distributions $y_t$ conditional on $y_0=\bar y$, where $\bar y$ is the
deterministic steady state. If triggered, Dynare++ simulates a given
number of samples with a given number of periods all starting at
the deterministic steady state. Then for each time $t$, mean
$E[y_t|y_0=\bar y]$ and variances $E[(y_t-E[y_t|y_0=\bar
y])(y_t-E[y_t|y_0=\bar y])^T|y_0=\bar y]$ are reported.

\subsubsection{Random Numbers}
\label{random_numbers}

For generating of the pseudo random numbers, Dynare++ uses Mersenne
twister by Makoto Matsumoto and Takuji Nishimura. Because of the
parallel nature of Dynare++ simulations, each simulated sample gets
its own instance of the twister. Each such instance is seeded before
the simulations are started. This is to prevent additional randomness
implied by the operating system's thread scheduler to interfere with
the pseudo random numbers.

For seeding the individual instances of the Mersenne twister assigned
to each simulated sample the system (C library) random generator is
used. These random generators do not have usually very good
properties, but we use them only to seed the Mersenne twister
instances. The user can set the initial seed of the system random
generator and in this way deterministically choose the seeds of all
instances of the Mersenne twister.

In this way, it is guaranteed that two runs of Dynare++
with the same seed will yield the same results regardless the
operating system's scheduler. The only difference may be caused by a
different round-off errors committed when the same set of samples are
summed in the different order (due to the operating system's scheduler).

\subsection{Numerical Approximation Checks}
\label{checks}

Optionally, Dynare++ can run three kinds of checks for Taylor
approximation errors. All three methods numerically calculate
the residual of the DSGE equations
\[
E[f(g^{**}(g^*(y^*,u),u'),g(y^*,u),y^*,u)|y^*,u]
\]
which must be ideally zero for all $y^*$ and $u$. This integral is
evaluated by either product or Smolyak rule applied to one dimensional
Gauss--Hermite quadrature. The user does not need to care about the
decision. An algorithm yielding higher quadrature level and less
number of evaluations less than a user given maximum is selected.

The three methods differ only by a set of $y^*$ and $u$ where the
residuals are evaluated. These are:
\begin{itemize}
\item The first method calculates the residuals along the shocks for
fixed $y^*$ equal to the fix point. We let all elements of $u$ be
fixed at $0$ but one element, which varies from $-\mu\sigma$ to
$\mu\sigma$, where $\sigma$ is a standard error of the element and
$\mu$ is the user given multiplier. In this way we can see how the
approximation error grows if the fix point is disturbed by a shock of
varying size.
\item The second method calculates the residuals along a simulation
path. A random simulation is run, and at each point the residuals are
reported.
\item The third method calculates the errors on an ellipse of the
state variables $y^*$. The shocks $u$ are always zero. The ellipse is
defined as
\[\{Ax|\; \Vert x\Vert_2=\mu\},\]
where $\mu$ is a user given multiplier, and $AA^T=V$ for $V$ being a
covariance of endogenous variables based on the first order
approximation. The method calculates the residuals at low discrepancy
sequence of points on the ellipse. Both the residuals and the points
are reported.
\end{itemize}

\section{Optimal Policy with Dynare++}
\label{optim}

Starting with version 1.3.2, Dynare++ is able to automatically
generate and then solve the first order conditions for a given
objective and (possibly) forward looking constraints. Since the
constraints can be forward looking, the use of this feature will
mainly be in optimal policy or control.

The only extra thing which needs to be added to the model file is a
specification of the policy's objective. This is done by two keywords,
placed not before parameter settings. If the objective is to maximize
$$E_{t_0}\sum_{t=t_0}^\infty\beta^{t-t_0}\left[\frac{c_t^{1-\eta}}{1-\eta}+
a\frac{g_t^{1-\phi}}{1-\phi}\right],$$
then the keywords will be:
{\small
\begin{verbatim}
planner_objective C^(1-eta)/(1-eta) + a*G^(1-phi)/(1-phi);

planner_discount beta;
\end{verbatim}
}

Dynare++ parses the file and if the two keywords are present, it
automatically derives the first order conditions for the problem. The
first order conditions are put to the form \eqref{focs} and solved. In
this case, the equations in the {\tt model} section are understood as
the constraints (they might come as the first order conditions from
optimizations of other agents) and their number must be less than the
number of endogenous variables.

This section further describes how the optimal policy first order
conditions look like, then discusses some issues with the initial
guess for deterministic steady state, and finally describes how to
simulate Ramsey policy within this framework.

\subsection{First Order Conditions}

Mathematically, the optimization problem looks as follows:
\begin{align}
\max_{\left\{y_\tau\right\}^\infty_t}&E_t
\left[\sum_{\tau=t}^\infty\beta^{\tau-t}b(y_{\tau-1},y_\tau,y_{\tau+1},u_\tau)\right]\notag\\
&\rm{s.t.}\label{planner_optim}\\
&\hskip1cm E^I_\tau\left[f(y_{\tau-1},y_\tau,y_{\tau+1},u_\tau)\right]=0\quad\rm{for\ }
\tau=\ldots,t-1,t,t+1,\ldots\notag
\end{align}
where $E^I$ is an expectation operator over an information set including,
besides all the past, all future realizations of policy's control
variables and distributions of future shocks $u_t\sim
N(0,\Sigma)$. The expectation operator $E$ integrates over an
information including only distributions of $u_t$ (besides the past).

Note that the constraints $f$ take place at all times, and they are
conditioned at the running $\tau$ since the policy knows that the
agents at time $\tau$ will use all the information available at
$\tau$.

The maximization problem can be rewritten using Lagrange multipliers as:
\begin{align}
\max_{y_t}E_t&\left[\sum_{\tau=t}^\infty\beta^{\tau-t}b(y_{\tau-1},y_\tau,y_{\tau+1},u_\tau)\right.\notag\\
&\left.+\sum_{\tau=-\infty}^{\infty}\beta^{\tau-t}\lambda^T_\tau E_\tau^I\left[f(y_{\tau-1},y_\tau,y_{\tau+1},u_\tau)\right]\right],
\label{planner_optim_l}
\end{align}
where $\lambda_t$ is a column vector of Lagrange multipliers.

After some manipulations with compounded expectations over different
information sets, one gets the following first order conditions:
\begin{align}
E_t\left[\vphantom{\frac{\int^(_)}{\int^(\_)}}\right.&\frac{\partial}{\partial y_t}b(y_{t-1},y_t,y_{t+1},u_t)+
\beta L^{+1}\frac{\partial}{\partial y_{t-1}}b(y_{t-1},y_t,y_{t+1},u_t)\notag\\
&+\beta^{-1}\lambda_{t-1}^TL^{-1}\frac{\partial}{\partial y_{t+1}}f(y_{t-1},y_t,y_{t+1},u_t)\notag\\
&+\lambda_t^T\frac{\partial}{\partial y_{t}}f(y_{t-1},y_t,y_{t+1},u_t)\notag\\
&+\beta\lambda_{t+1}^TE_{t+1}\left[L^{+1}\frac{\partial}{\partial y_{t-1}}f(y_{t-1},y_t,y_{t+1},u_t)\right]
\left.\vphantom{\frac{\int^(_)}{\int^(\_)}}\right]
 = 0,\label{planner_optim_foc2}
\end{align}
where $L^{+1}$ is one period lead operator, and $L^{-1}$ is one period lag operator.

Dynare++ takes input corresponding to \eqref{planner_optim},
introduces the Lagrange multipliers according to
\eqref{planner_optim_l}, and using its symbolic derivator it compiles
\eqref{planner_optim_foc2}. The system \eqref{planner_optim_foc2} with
the constraints from \eqref{planner_optim_l} is then solved in the
same way as the normal input \eqref{focs}.

\subsection{Initial Guess for Deterministic Steady State}
\label{opt_init}

Solving deterministic steady state of non-linear dynamic systems is
not trivial and the first order conditions for optimal policy add
significant complexity. The {\tt initval} section allows to input the
initial guess of the non-linear solver. It requires that all user
declared endogenous variables be initialized. However, in most cases,
we have no idea what are good initial guesses for the Lagrange
multipliers.

For this reason, Dynare++ calculates an initial guess of Lagrange
multipliers using user provided initial guesses of all other
endogenous variables. It uses the linearity of the Lagrange
multipliers in the \eqref{planner_optim_foc2}. In its static form,
\eqref{planner_optim_foc2} looks as follows:
\begin{align}
&\frac{\partial}{\partial y_t}b(y,y,y,0)+
\beta\frac{\partial}{\partial y_{t-1}}b(y,y,y,0)\notag\\
&+\lambda^T\left[\beta^{-1}\frac{\partial}{\partial y_{t+1}}f(y,y,y,0)
 +\frac{\partial}{\partial y_{t}}f(y,y,y,0)
 +\beta\frac{\partial}{\partial y_{t-1}}f(y,y,y,0)\right]
 = 0\label{planner_optim_static}
\end{align}

The user is required to provide an initial guess of all declared
variables (all $y$). Then \eqref{planner_optim_static} becomes an
overdetermined linear system in $\lambda$, which is solved by means of
the least squares. The closer the initial guess of $y$ is to the exact
solution, the closer are the Lagrange multipliers $\lambda$.

The calculated Lagrange multipliers by the least squares are not used,
if they are set in the {\tt initval} section. In other words, if a
multiplier has been given a value in the {\tt initval} section, then
the value is used, otherwise the calculated value is taken.

For even more difficult problems, Dynare++ generates two Matlab files
calculating a residual of the static system and its derivative. These
can be used in Matlab's {\tt fsolve} or other algorithm to get an
exact solution of the deterministic steady state. See
\ref{output_matlab_scripts} for more details.

Finally, Dynare++ might generate a few auxiliary variables. These are
simple transformations of other variables. They are initialized
automatically and the user usually does not need to care about it.

\subsection{Optimal Ramsey Policy}
\label{ramsey}

Dynare++ solves the optimal policy problem with timeless
perspective. This means that it assumes that the constraints in
\eqref{planner_optim} are valid from the infinite past to infinite
future. Dynare++ calculation of ergodic distribution then assumes that
the policy has been taking place from infinite past.

If some constraints in \eqref{planner_optim} are forward looking, this
will result in some backward looking Lagrange multipliers. Such
multipliers imply possibly time inconsistent policy in the states of
the ``original'' economy, since these backward looking multipliers add
new states to the ``optimized'' economy. In this respect, the timeless
perspective means that there is no fixed initial distribution of such
multipliers, instead, their ergodic distribution is taken.

In contrast, Ramsey optimal policy is started at $t=0$. This means
that the first order conditions at $t=0$ are different than the first
order conditions at $t\geq 1$, which are
\eqref{planner_optim_foc2}. However, it is not difficult to assert
that the first order conditions at $t=0$ are in the form of
\eqref{planner_optim_foc2} if all the backward looking Lagrange
multipliers are set to zeros at period $-1$, i.e. $\lambda_{-1}=0$.

All in all, the solution of \eqref{planner_optim_foc2} calculated by
Dynare++ can be used as a Ramsey optimal policy solution provided that
all the backward looking Lagrange multipliers were set to zeros prior
to the first simulation period. This can be done by setting the
initial state of a simulation path in {\tt dynare\_simul.m}. If this
is applied on the example from \ref{optim_tut}, then we may do the
following in the command prompt:
{\small
\begin{verbatim}
>> load kp1980_2.mat
>> shocks = zeros(1,100);
>> ystart = dyn_ss;
>> ystart(dyn_i_MULT3) = 0;
>> r=dynare_simul('kp1980_2.mat',shocks,ystart);
\end{verbatim}
}
This will simulate the economy if the policy was introduced in the
beginning and no shocks happened.

More information on custom simulations can be obtained by typing:
{\small
\begin{verbatim}
help dynare_simul
\end{verbatim}
}


\section{Running Dynare++}

This section deals with Dynare++ input. The first subsection
\ref{dynpp_opts} provides a list of command line options, next
subsection \ref{dynpp_mod} deals with a format of Dynare++ model file,
and the last subsection discusses incompatibilities between Dynare
Matlab and Dynare++.

\subsection{Command Line Options}
\label{dynpp_opts}

The calling syntax of the Dynare++ is

{\small
\begin{verbatim}
dynare++ [--help] [--version] [options] <model file>
\end{verbatim}
}

\noindent where the model file must be given as the last token and
must include its extension. The model file may include path, in this
case, the path is taken relative to the current directory. Note that
the current directory can be different from the location of {\tt
dynare++} binary.

The options are as follows:

\def\desc#1{\rlap{#1}\kern4cm}

\begin{description}
\item[\desc{\tt --help}] This prints a help message and exits.

\item[\desc{\tt --version}] This prints a version information and
exits.

\item[\desc{\tt --per \it num}] This sets a number of simulated
periods to {\it num}. This number is used when calculating
unconditional mean and covariance and for IRFs. Default is 100.

\item[\desc{\tt --sim \it num}] This sets a number of stochastic
simulations. This number is used when calculating unconditional mean
and covariance and for IRFs. The total sample size for unconditional
mean and covariance is the number of periods times the number of
successful simulations. Note that if a simulation results in {\tt NaN}
or {\tt +-Inf}, then it is thrown away and is not considered for the
mean nor the variance. The same is valid for IRF. Default is 80.

\item[\desc{\tt --rtsim \it num}] This sets a number of stochastic
simulations whose statistics are calculated in the real-time. See
\ref{rt_simul} for more details. Default is 0, no simulations.

\item[\desc{\tt --rtper \it num}] This sets a number of simulated
periods per one simulation with real-time statistics to {\it num}. See
\ref{rt_simul} for more details. Default is 0, no simulations.

\item[\desc{\tt --condsim \it num}] This sets a number of stochastic
conditional simulations. See \ref{cond_dist} for more details. Default
is 0, no simulations.

\item[\desc{\tt --condper \it num}] This sets a number of simulated
periods per one conditional simulation. See \ref{cond_dist} for more
details. Default is 0, no simulations.

\item[\desc{\tt --steps \it num}] If the number {\it num} is greater
than 0, this option invokes a multi-step algorithm (see section
\ref{dynpp_calc}), which in the given number of steps calculates fix
points and approximations of the decision rule for increasing
uncertainty. Default is 0, which invokes a standard algorithm for
approximation about deterministic steady state. For more details,
see \ref{multistep_alg}.

\item[\desc{\tt --centralize}] This option causes that the resulting
decision rule is centralized about (in other words: expressed in the
deviations from) the stochastic fix point. The centralized decision
rule is mathematically equivalent but has an advantage of yielding
less numerical errors in average than not centralized decision
rule. By default, the rule is centralized. For more details, see
\ref{dr_form}.

\item[\desc{\tt --no-centralize}] This option causes that the
resulting decision rule is not centralized about (in other words:
expressed in the deviations from) the stochastic fix point. By
default, the rule is centralized. For more details, see
\ref{dr_form}.

This option has no effect if the number of steps given by {\tt
--steps} is greater than 0. In this case, the rule is always
centralized.

\item[\desc{\tt --prefix \it string}] This sets a common prefix of
variables in the output MAT file. Default is {\tt dyn}.

\item[\desc{\tt --seed \it num}] This sets an initial seed for the
random generator providing seed to generators for each sample. See
\ref{random_numbers} for more details. Default is 934098.

\item[\desc{\tt --order \it num}] This sets the order of approximation
and overrides the {\tt order} statement in the model file. There is no
default.

\item[\desc{\tt --threads \it num}] This sets a number of parallel
threads. Complex evaluations of Faa Di Bruno formulas, simulations and
numerical integration can be parallelized, Dynare++ exploits this
advantage. You have to have a hardware support for this, otherwise
there is no gain from the parallelization. As a rule of thumb, set the
number of threads to the number of processors. An exception is a
machine with Pentium 4 with Hyper Threading (abbreviated by HT). This
processor can run two threads concurrently. The same applies to
Dual-Core processors. Since these processors are present in most new
PC desktops/laptops, the default is 2.

\item[\desc{\tt --ss-tol \it float}] This sets the tolerance of the
non-linear solver of deterministic steady state to {\it float}. It is
in $\Vert\cdot\Vert_\infty$ norm, i.e. the algorithm is considered as
converged when a maximum absolute residual is less than the
tolerance. Default is $10^{-13}$.

\item[\desc{\tt --check \it pPeEsS}] This selects types of residual
checking to be performed. See section \ref{checks} for details. The
string consisting of the letters ``pPeEsS'' governs the selection. The
upper-case letters switch a check on, the lower-case letters
off. ``P'' stands for checking along a simulation path, ``E'' stands
for checking on ellipse, and finally ``S'' stands for checking along
the shocks. It is possible to choose more than one type of check. The
default behavior is that no checking is performed.

\item[\desc{\tt --check-evals \it num}] This sets a maximum number of
evaluations per one re\-sidual. The actual value depends on the selected
algorithm for the integral evaluation. The algorithm can be either
product or Smolyak quadrature and is chosen so that the actual number
of evaluations would be minimal with maximal level of
quadrature. Default is 1000.

\item[\desc{\tt --check-num \it num}] This sets a number of checked
points in a residual check. One input value $num$ is used for all
three types of checks in the following way:
\begin{itemize}
\item For checks along the simulation, the number of simulated periods
is $10\cdot num$
\item For checks on ellipse, the number of points on ellipse is $10\cdot num$
\item For checks along the shocks, the number of checked points
corresponding to shocks from $0$ to $\mu\sigma$ (see \ref{checks}) is
$num$.
\end{itemize}
Default is 10.

\item[\desc{\tt --check-scale \it float}] This sets the scaling factor
$\mu$ for checking on ellipse to $0.5\cdot float$ and scaling factor
$\mu$ for checking along shocks to $float$. See section
\ref{checks}. Default is 2.0.

\item[\desc{\tt --no-irfs}] This suppresses IRF calculations. Default
is to calculate IRFs for all shocks.

\item[\desc{\tt --irfs}] This triggers IRF calculations. If there are
no shock names following the {\tt --irfs} option, then IRFs for all
shocks are calculated, otherwise see below. Default is to calculate
IRFs for all shocks.

\item[\desc{\tt --irfs \it shocklist}] This triggers IRF calculations
only for the listed shocks. The {\it shocklist} is a space separated
list of exogenous variables for which the IRFs will be
calculated. Default is to calculate IRFs for all shocks.
\end{description}

The following are a few examples:
{\small
\begin{verbatim}
dynare++ --sim 300 --per 50 blah.mod
dynare++ --check PE --check-num 15 --check-evals 500 blah.dyn
dynare++ --steps 5 --check S --check-scale 3 blahblah.mod
\end{verbatim}
}
The first one sets the number of periods for IRF to 50, and sets a sample
size for unconditional mean and covariance calculations to 6000. The
second one checks the decision rule along a simulation path having 150
periods and on ellipse at 150 points performing at most 500 evaluations
per one residual. The third one solves the model in five steps and
checks the rule along all the shocks from $-3\sigma$ to $3\sigma$ in
$2*10+1$ steps (10 for negative, 10 for positive and 1 for at zero).

\subsection{Dynare++ Model File}
\label{dynpp_mod}

In its strictest form, Dynare++ solves the following mathematical problem:
\begin{equation}\label{basic_form}
E_t[f(y^{**}_{t+1},y_t,y^*_{t-1},u_t)]=0
\end{equation}
This problem is input either directly, or it is an output of Dynare++
routines calculating first order conditions of the optimal policy
problem. In either case, Dynare++ performs necessary and
mathematically correct substitutions to put the user specified problem
to the \eqref{basic_form} form, which goes to Dynare++ solver. The
following discusses a few timing issues:
\begin{itemize}
\item Endogenous variables can occur, starting from version 1.3.4, at
times after $t+1$. If so, an equation containing such occurrence is
broken to non-linear parts, and new equations and new auxiliary
variables are automatically generated only for the non-linear terms
containing the occurrence. Note that shifting such terms to time $t+1$
may add occurrences of some other variables (involved in the terms) at
times before $t-1$ implying addition of auxiliary variables to bring
those variables to $t-1$.
\item Variables declared as shocks may occur also at arbitrary
times. If before $t$, additional endogenous variables are used to
bring them to time $t$. If after $t$, then similar method is used as
for endogenous variables occurring after $t+1$.
\item There is no constraint on variables occurring at both times
$t+1$ (or later) and $t-1$ (or earlier). Virtually, all variables can
occur at arbitrary times.
\item Endogenous variables can occur at times before $t-1$. If so,
additional endogenous variables are added for all lags between the
variable and $t-1$.
\item Dynare++ applies the operator $E_t$ to all occurrences at time
$t+1$. The realization of $u_t$ is included in the information set of
$E_t$. See an explanation of Dynare++ timing on page \pageref{timing}.
\end{itemize}

The model equations are formulated in the same way as in Matlab
Dynare. The time indexes different from $t$ are put to round
parenthesis in this way: {\tt C(-1)}, {\tt C}, {\tt C(+1)}.

The mathematical expressions can use the following functions and operators:
\begin{itemize}
\item binary {\tt + - * / \verb|^|}
\item unary plus and minus minus as in {\tt a = -3;} and {\tt a = +3;} resp.
\item unary mathematical functions: {\tt log exp sin cos tan
sqrt}, whe\-re the logarithm has a natural base
\item symbolic differentiation operator {\tt diff(expr,symbol)}, where
{\tt expr} is a mathematical expression and {\tt symbol} is a unary
symbol (a variable or a parameter); for example {\tt
  diff(A*K(-1)\verb|^|alpha*L\verb|^|(1-alpha),K(-1))} is internally expanded as
{\tt A*alpha*K(-1)\verb|^|(alpha-1)*L\verb|^|(1-alpha)}
\item unary error function and complementary error function: {\tt erf}
and {\tt erfc} defined as
\begin{eqnarray*}
erf(x) &= \frac{2}{\sqrt{\pi}}\int_0^x e^{-t^2}{\rm d}t\\
erfc(x)&= \frac{2}{\sqrt{\pi}}\int_x^\infty e^{-t^2}{\rm d}t
\end{eqnarray*}
\end{itemize}

The model file can contain user comments. Their usage can be
understood from the following piece of the model file:

{\small
\begin{verbatim}
P*C^(-gamma) = // line continues until semicolon
  beta*C(+1)^(-gamma)*(P(+1)+Y(+1)); // asset price
// choose dividend process: (un)comment what you want
Y/Y_SS = (Y(-1)/Y_SS)^rho*exp(EPS);
/*
Y-Y_SS = rho*(Y(-1)-Y_SS)+EPS;
*/
\end{verbatim}
}

\subsection{Incompatibilities with Matlab Dynare}

This section provides a list of incompatibilities between a model file
for Dy\-na\-re++ and Matlab Dynare. These must be considered when a model
file for Matlab Dynare is being migrated to Dynare++. The list is the
following:
\begin{itemize}
\item There is no {\tt periods} keyword.
\item The parameters cannot be lagged or leaded, I think that Dynare
Matlab allows it, but the semantics is the same (parameter is a
constant).
\item There are no commands like {\tt steady}, {\tt check}, {\tt
simul}, {\tt stoch\_simul}, etc.
\item There are no sections like {\tt estimated\_params}, {\tt
var\_obs}, etc.
\item The variance-covariance matrix of endogenous shocks is given by
{\tt vcov} matrix in Dynare++. An example follows. Starting from
version 1.3.5, it is possible for vcov to be positive semi-definite
matrix.
{\small
\begin{verbatim}
vcov = [
0.05 0 0 0;
0 0.025 0 0;
0 0 0.05 0;
0 0 0 0.025
];
\end{verbatim}
}

\end{itemize}

\section{Dynare++ Output}

There are three output files; a data file in MAT-4 format containing
the output data (\ref{matfile}), a journal text file containing an
information about the Dynare++ run (\ref{journalfile}), and a dump
file (\ref{dumpfile}). Further, Dynare++ generates two Matlab script
files, which calculate a residual and the first derivative of the
residual of the static system (\ref{output_matlab_scripts}). These are
useful when calculating the deterministic steady state outside
Dynare++.

Note that all output files are created in the current directory of
the Dynare++ process. This can be different from the directory where
the Dynare++ binary is located and different from the directory where
the model file is located.

Before all, we need to understand what variables are automatically
generated in Dynare++.

\subsection{Auxiliary Variables}
\label{aux_var}

Besides the endogenous variables declared in {\tt var} section,
Dynare++ might automatically add the following endogenous variables:

\halign{\vrule width0pt height14pt{\tt #}\hfil & \kern 3mm%
\vtop{\rightskip=0pt plus 5mm\noindent\hsize=9.5cm #}\cr
MULT{\it n}& A Lagrange multiplier of the optimal policy problem
associated with a constraint number {\it n} starting from zero.\cr
AUX\_{\it n1}\_{\it n2}\_{\it n3}& An auxiliary variable associated
with the last term in equation \eqref{planner_optim_foc2}. Since the
term is under $E_{t+k}$, we need the auxiliary variable be put back
in time. {\it n1} is a variable number starting from 0 in the declared
order with respect to which the term was differentiated, {\it n2} is a
number of constraint starting from 0, and finally {\it n3} is $k$
(time shift of the term).\cr
{\it endovar}\_p{\it K}& An auxiliary variable for bringing an
endogenous variable {\it endovar} back in time by $K$ periods. The
semantics of this variables is {\tt {\it endovar}\_p{\it K} = {\it
endovar}(+{\it K})}.\cr
{\it endovar}\_m{\it K}& An auxiliary variable for bringing an
endogenous variable {\it endovar} forward in time by $K$ periods. The
semantics of this variables is {\tt {\it endovar}\_m{\it K} = {\it
endovar}(-{\it K})}.\cr
{\it exovar}\_e& An auxiliary endogenous variable made equal to the
exogenous variable to allow for a semantical occurrence of the
exogenous variable at time other than $t$. The semantics of this
variables is {\tt {\it exovar}\_e = {\it exovar}}.\cr
AUXLD\_{\it n1}\_{\it n2}\_{\it n3}& An auxiliary variable for
bringing a non-linear term containing an occurrence of a variable
after $t+1$ to time $t+1$. {\it n1} is an equation number starting
from 0, {\it n2} is the non-linear sub-term number in the equation
starting from 0. {\it n3} is a time shift. For example, if the first
equation is the following:
\begin{verbatim}
X - Y*W(+1) + W(+2)*Z(+4) = 0;
\end{verbatim}
then it will be expanded as:
\begin{verbatim}
X - Y*W(+1) + AUXLD_0_2_3(+1) = 0;
AUXLD_0_2_1 = W(-1)*Z(+1);
AUXLD_0_2_2 = AUXLD_0_2_1(+1);
AUXLD_0_2_3 = AUXLD_0_2_2(+1);
\end{verbatim}
\cr
}

\subsection{MAT File}
\label{matfile}

The contents of the data file is depicted below. We
assume that the prefix is {\tt dyn}.

\halign{\vrule width0pt height14pt{\tt #}\hfil & \kern 3mm%
\vtop{\rightskip=0pt plus 5mm\noindent\hsize=7.5cm #}\cr
dyn\_nstat& Scalar. A number of static variables
(those occurring only at time $t$).\cr
dyn\_npred & Scalar. A number of variables occurring
at time $t-1$ and not at $t+1$.\cr
dyn\_nboth & Scalar. A number of variables occurring
at $t+1$ and $t-1$.\cr
dyn\_nforw & Scalar. A number of variables occurring
at $t+1$ and not at $t-1$.\cr
dyn\_vars & Column vector of endogenous variable
names in Dy\-na\-re++ internal ordering.\cr
dyn\_i\_{\it endovar} & Scalar. Index of a variable
named {\it endovar} in the {\tt dyn\_vars}.\cr
dyn\_shocks & Column vector of exogenous variable
names.\cr
dyn\_i\_{\it exovar} & Scalar. Index of a shock
named {\it exovar} in the {\tt dyn\_shocks}.\cr
dyn\_state\_vars & Column vector of state variables,
these are stacked variables counted by {\tt dyn\_\-npred}, {\tt
dyn\_\-nboth} and shocks.\cr
dyn\_vcov\_exo & Matrix $nexo\times nexo$. The
variance-covariance matrix of exogenous shocks as input in the model
file. The ordering is given by {\tt dyn\_shocks}.\cr
dyn\_mean & Column vector $nendo\times 1$. The
unconditional mean of endogenous variables. The ordering is given by
{\tt dyn\_vars}.\cr
dyn\_vcov & Matrix $nendo\times nendo$. The
unconditional covariance of endogenous variables. The ordering is given
by {\tt dyn\_vars}.\cr
dyn\_rt\_mean & Column vector $nendo\times 1$. The unconditional mean
of endogenous variables estimated in real-time. See
\ref{rt_simul}. The ordering is given by {\tt dyn\_vars}.\cr
dyn\_rt\_vcov & Matrix $nendo\times nendo$. The unconditional
covariance of endogenous variables estimated in real-time. See \ref{rt_simul}. The
ordering is given by {\tt dyn\_vars}.\cr
dyn\_cond\_mean & Matrix $nendo\times nper$. The rows correspond to
endogenous variables in the ordering of {\tt dyn\_vars}, the columns
to periods. If $t$ is a period (starting with 1), then $t$-th column
is $E[y_t|y_0=\bar y]$. See \ref{cond_dist}.\cr
dyn\_cond\_variance & Matrix $nendo\times nper$. The rows correspond
to endogenous variables in the ordering of {\tt dyn\_vars}, the
columns to periods. If $t$ is a period (starting with 1), then $t$-th
column are the variances of $y_t|y_0=\bar y$. See \ref{cond_dist}.\cr
dyn\_ss & Column vector $nendo\times 1$. The fix
point of the resulting approximation of the decision rule.\cr
dyn\_g\_{\it order} & Matrix $nendo\times ?$. A
derivative of the decision rule of the {\it order} multiplied by
$1/order!$. The rows correspond to endogenous variables in the
ordering of {\tt dyn\_vars}. The columns correspond to a
multidimensional index going through {\tt dyn\_state\_vars}. The data
is folded (all symmetrical derivatives are stored only once).\cr
dyn\_steady\_states & Matrix $nendo\times
nsteps+1$. A list of fix points at which the multi-step algorithm
calculated approximations. The rows correspond to endogenous variables
and are ordered by {\tt dyn\_vars}, the columns correspond to the
steps. The first column is always the deterministic steady state.\cr
dyn\_irfp\_{\it exovar}\_mean & Matrix
$nendo\times nper$. Positive impulse response to a shock named {\it
exovar}. The row ordering is given by {\tt dyn\_vars}. The columns
correspond to periods.\cr
dyn\_irfp\_{\it exovar}\_var & Matrix
$nendo\times nper$. The variances of positive impulse response
functions.\cr
dyn\_irfm\_{\it exovar}\_mean & Same as {\tt
dyn\_irfp\_}{\it exovar}{\tt \_mean} but for negative impulse.\cr
dyn\_irfp\_{\it exovar}\_var & Same as {\tt
dyn\_irfp\_}{\it exovar}{\tt \_var} but for negative impulse.\cr
dyn\_simul\_points & A simulation path along which the check was
done. Rows correspond to endogenous variables, columns to
periods. Appears only if {\tt --check P}.\cr
dyn\_simul\_errors & Errors along {\tt
dyn\_simul\_points}. The rows correspond to equations as stated in the
model file, the columns to the periods. Appears only if {\tt --check
P}\cr
dyn\_ellipse\_points & A set of points on the ellipse at which the
approximation was checked. Rows correspond to state endogenous
variables (the upper part of {\tt dyn\_state\_vars}, this means
without shocks), and columns correspond to periods. Appears only if
{\tt --check E}.\cr
dyn\_ellipse\_errors & Errors on the ellipse points {\tt
dyn\_ellipse\_points}. The rows correspond to the equations as stated
in the model file, columns to periods. Appears only if {\tt --check
E}.\cr
dyn\_shock\_{\it exovar}\_errors& Errors along a shock named {\it
exovar}. The rows correspond to the equations as stated in the model
file. There are $2m+1$ columns, the middle column is the error at zero
shock. The columns to the left correspond to negative values, columns
to the right to positive. Appears only if {\tt --check S}.\cr
}

\subsection{Journal File}
\label{journalfile}

The journal file provides information on resources usage during the
run and gives some informative messages. The journal file is a text
file, it is organized in single line records. The format of records is
documented in a header of the journal file.

The journal file should be consulted in the following circumstances:
\begin{itemize}
\item Something goes wrong. For example, if a model is not
Blanchard--Kahn stable, then the eigenvalues are dumped to the journal
file.

If the unconditional covariance matrix {\tt dyn\_vcov} is NaN, then
from the journal file you will know that all the simulations had to be
thrown away due to occurrence of NaN or Inf. This is caused by
non-stationarity of the resulting decision rule.

If Dynare++ crashes, the journal file can be helpful for guessing a
point where it crashed.

\item You are impatient. You might be looking at the journal file
during the run in order to have a better estimate about the time when
the calculations are finished. In Unix, I use a command {\tt tail -f
blah.jnl}.\footnote{This helps to develop one of the three
programmer's virtues: {\it impatience}. The other two are {\it
laziness} and {\it hubris}; according to Larry Wall.}

\item Heavy swapping. If the physical memory is not
sufficient, an operating system starts swapping memory pages with a
disk. If this is the case, the journal file can be consulted for
information on memory consumption and swapping activity.

\item Not sure what Dynare++ is doing. If so, read the journal file,
which contains a detailed record on what was calculated, simulated
etc.
\end{itemize}

\subsection{Dump File}
\label{dumpfile}

The dump file is always created with the suffix {\tt .dump}. It is a
text file which takes a form of a model file. It sets the parameter
values which were used, it has the initval section setting the values
which were finally used, and mainly it has a model section of all
equations with all substitutions and formed the first order conditions
of the planner.

The dump file serves for debugging purposes, since it contains the
mathematical problem which is being solved by dynare++.

\subsection{Matlab Scripts for Steady State Calculations}
\label{output_matlab_scripts}

This section describes two Matlab scripts, which are useful when
calculating the deterministic steady state outside Dynare++. The
scripts are created by Dynare++ as soon as an input file is parsed,
that is before any calculations.

The first Matlab script having a name {\tt {\it modname}\_f.m} for
given parameters values and given all endogenous variables $y$
calculates a residual of the static system. Supposing the model is in
the form of \eqref{focs}, the script calculates a vector:
\[
f(y,y,y,0)
\]

The second script having a name {\tt {\it modname}\_ff.m} calculates a matrix:
\[
\frac{\partial}{\partial y}f(y,y,y,0)
\]

Both scripts take two arguments. The first is a vector of parameter
values ordered in the same ordering as declared in the model file. The
second is a vector of all endogenous variables at which the evaluation
is performed. These endogenous variables also include auxiliary
variables automatically added by Dynare++ and Lagrange multipliers if
an optimal policy problem is solved. If no endogenous variable has not
been added by Dynare++, then the ordering is the same as the ordering
in declaration in the model file. If some endogenous variables have
been added, then the ordering can be read from comments close to the
top of either two files.

For example, if we want to calculate the deterministic steady state of
the {\tt kp1980.dyn} model, we need to do the following:
\begin{enumerate}
\item Run Dynare++ with {\tt kp1980.dyn}, no matter if the calculation
has not been finished, important output are the two Matlab scripts
created just in the beginning.
\item Consult file {\tt kp1980\_f.m}\ to get the ordering of parameters
and all endogenous variables.
\item Create a vector {\tt p} with the parameter values in the ordering
\item Create a vector {\tt init\_y} with the initial guess for the
Matlab solver {\tt fsolve}
\item Create a simple Matlab function called {\tt kp1980\_fsolve.m}\ 
returning the residual and Jacobian:
{\small
\begin{verbatim}
function [r, J] = kp1980_fsolve(p, y)
  r = kp1980_f(p, y);
  J = kp1980_ff(p, y);
\end{verbatim}
}
\item In the Matlab prompt, run the following:
{\small
\begin{verbatim}
opt=optimset('Jacobian','on','Display','iter');
y=fsolve(@(y) kp1980_fsolve(p,y), init_y, opt);
\end{verbatim}
}
\end{enumerate}
 

\subsection{Custom Simulations}
\label{custom}

When Dynare++ run is finished it dumps the derivatives of the
calculated decision rule to the MAT file. The derivatives can be used
for a construction of the decision rule and custom simulations can be
run. This is done by {\tt dynare\_simul.m} M-file in Matlab. It reads
the derivatives and simulates the decision rule with provided
realization of shocks.

All the necessary documentation can be viewed by the command:
{\small
\begin{verbatim}
help dynare_simul
\end{verbatim}
}

\end{document}
