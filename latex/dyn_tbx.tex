% ----------------------------------------------------------------
% AMS-LaTeX Paper ************************************************
% **** -----------------------------------------------------------
\documentclass[12pt,a4paper]{article}
\usepackage{amssymb,amsmath}
\usepackage[dvips]{graphicx}
\usepackage{natbib}
\usepackage{psfrag}
\usepackage{setspace}
\usepackage{rotating}
%\singlespacing (interlinea singola)
%\onehalfspacing (interlinea 1,5)
%\doublespacing (interlinea doppia)


%\bibpunct{(}{)}{;}{a}{,}{,}
\bibpunct[, ]{(}{)}{;}{a}{,}{,}
%\pagestyle{headings}
% ----------------------------------------------------------------
\begin{document}

% ----------------------------------------------------------------
\title{Sensitivity Analysis Toolbox for DYNARE}%

\author{Marco Ratto\\
European Commission, Joint Research Centre \\
TP361, IPSC, via E. Fermi 1\\21020 Ispra
(VA) Italy\\
\texttt{marco.ratto@jrc.it}
\thanks{The author gratefully thanks Christophe Planas, Kenneth Judd, Michel Juillard,
Alessandro Rossi, Frank Schorfheide and the participants to the
Courses on Global Sensitivity Analysis for Macroeconomic
Models (Ispra, 2006-2007-2008) for interesting discussions and
helpful suggestions.}}

%%% To have the current date inserted, use \date{\today}:
%%% To insert a footnote, add thanks in the date/title/author fields:
\date{\today}
%\date{\today \thanks{Authors gratefully acknowledge the
%contribution by ... for ...}}
\maketitle %\tableofcontents

%\doublespacing

%-----------------------------------------------------------------------
\begin{abstract}
\noindent The Sensitivity Analysis Toolbox for DYNARE is a set of
MATLAB routines for the analysis of DSGE models with global
sensitivity analysis. The routines are thought to be used within
the DYNARE v4 environment.


\begin{description}
  \item \textbf{Keywords}: Stability Mapping , Reduced form solution, DSGE models,
  Monte Carlo filtering,   Global Sensitivity Analysis.
\end{description}
\end{abstract}
\newpage
% ----------------------------------------------------------------
\section{Introduction} \label{s:intro}
The Sensitivity Analysis Toolbox for DYNARE is a collection of
MATLAB routines implemented to answer the following questions: (i)
Which is the domain of structural coefficients assuring the
stability and determinacy of a DSGE model? (ii) Which parameters
mostly drive the fit of, e.g., GDP and which the fit of inflation?
Is there any conflict between the optimal fit of one observed
series versus another one? (iii) How to represent in a direct,
albeit approximated, form the relationship between structural
parameters and the reduced form of a rational expectations model?

The discussion of the methodologies and their application is
described in \cite{Ratto_CompEcon_2008}.


\section{Use of the Toolbox}
The DYNARE parser now recognizes sensitivity analysis commands.
The syntax is based on a single command:
\vspace{0.5cm}

\verb"dynare_sensitivity(option1=<opt1_val>,option2=<opt2_val>,...)"

\vspace{0.5cm} \noindent with a list of options described in the
next section.

With respect to the previous version of the toolbox, in order to
work properly, the sensitivity analysis Toolbox \emph{no longer}
needs that the DYNARE estimation environment is set-up.

Therefore, \verb"dynare_sensitivity" is the only command to run to
make a sensitivity analysis on a DSGE model\footnote{Of course,
when the user needs to perform the mapping of the fit with a
posterior sample, a Bayesian estimation has to be performed
beforehand}.


\section{List of options}

\subsection{Sampling options}
\begin{tabular}{r|l|l}
  % after \\ : \hline or \cline{col1-col2} \cline{col3-col4} ...
                option name & default & description  \\ \hline
                 \verb"Nsam"& 2048& Size of MC sample \\
               \verb"ilptau"& 1& 1 = use $LP_\tau$ quasi-Monte Carlo \\
                            &  & 0 = use LHS Monte Carlo \\
               \verb"pprior"& 1& 1 = sample from prior distributions\\
                            &  & 0 = sample from multivariate
                            normal \\
                            &  & \hspace{0.5 cm} $N(\hat{\theta},\Sigma)$, $\hat{\theta}$ is posterior mode  \\
                            &  & \hspace{0.5 cm} $\Sigma = H^{-1}$, $H$ is Hessian at the mode\\
          \verb"prior_range"& 1& 1 = sample \textit{uniformly} from prior ranges\\
                            &  & 0 = sample from prior  distributions: \\
                            &  & this requires MATLAB Statistics Toolbox\\
               \verb"morris"& 0& 0 = no Morris sampling for screening \\
                            &  & 1 = Morris sampling for screening     \\
          \verb"morris_nliv"& 6& number of levels in Morris design\\
          \verb"morris_ntra"& 20& number of trajectories in Morris design\\
                \verb"ppost"& 0& 0 = don't use Metropolis posterior sample\\
                            &  & 1 = use Metropolis posterior sample: this \\
                            &  & \hspace{0.5 cm} overrides any other sampling option!  \\ \hline
\end{tabular}
\subsection{Stability mapping}
\begin{tabular}{r|l|l}
                option name & default & description  \\ \hline
                 \verb"stab"& 1& 1 = perform stability mapping \\
                            &  & 0 = no stability mapping is performed\\
            \verb"load_stab"& 0& 0 =  generate a new sample\\
                            &  & 1 = load a previously created sample \\
          \verb"alpha2_stab"& 0.4& critical value for correlations $\rho$ in filtered samples:\\
                            &    & plot couples of parameters with \\
                            &   & $|\rho|>$\verb"alpha2_stab"\\
               \verb"ksstat"& 0.1& critical value for Smirnov statistics $d$: \\
                            &   & plot parameters with $d>$\verb"ksstat"\\ \hline
\end{tabular}

\newpage
\subsection{Reduced form mapping}% and identification}
The mapping of the reduced form solution forces the use of samples
from prior ranges or prior distributions, i.e.:
\\
\verb"options_.opt_gsa.pprior=1;"\\
\verb"options_.opt_gsa.ppost=0;"\\

It uses 250 samples to optimize smoothing parameters and 1000
samples to compute the fit. The rest of the sample is used  for
out-of-sample validation. \vspace{0.5cm}


\begin{tabular}{r|l|l}
                option name & default & description  \\ \hline
              \verb"redform"& 0& 0 = don't prepare MC sample of \\
                            &  & reduced form matrices \\
                            &  & 1 = prepare MC sample of \\
                            &  & reduced form matrices \\
         \verb"load_redform"& 0& 0 = estimate the mapping of \\
                            &  & reduced form model\\
                            &  & 1 = load previously estimated mapping\\
     \verb"logtrans_redform"& 0& 0 = use raw entries\\
                            &  & 1 = use log-transformed entries \\
    \verb"threshold_redform"& []& [] = don't filter MC entries \\
                            &   & of reduced form coefficients\\
                            &   & [\verb"max" \verb"max"] =  analyse filtered \\
                            &   & entries within the range [\verb"max" \verb"max"] \\
       \verb"ksstat_redform"& 0.1& critical value for Smirnov statistics $d$ \\
                            &   & when reduced form entries are filtered\\
       \verb"alpha2_redform"& 0.3& critical value for correlation $\rho$ \\
                            &   & when reduced form entries are filtered\\
              \verb"namendo"& [] & list of endogenous variables \\
                            & : & jolly character to indicate ALL endogenous \\
           \verb"namlagendo"& [] & list of lagged endogenous variables:\\
                            &   & analyse entries [\verb"namendo"$\times$\verb"namlagendo"]\\
                            & : & jolly character to indicate ALL endogenous \\
               \verb"namexo"& []& list of exogenous variables:\\
                            &   & analyse entries
                            [\verb"namendo"$\times$\verb"namexo"]\\
                            & : & jolly character to indicate ALL exogenous  \\\hline
\end{tabular}
\vspace{0.5cm} \\

One can also load a previously estimated mapping
with a new MC sample, to look at the forecast for the new MC sample.\\

\subsection{Mapping the fit}
The RMSE analysis can be performed with different types of
sampling options:
\begin{enumerate}
\item when \verb"pprior=1" and \verb"ppost=0", the Toolbox analyses the RMSE's for
the MC sample obtained by sampling parameters from their prior
distributions (or prior ranges): this analysis provides some hints
about what parameter drives the fit of which observed series,
prior to the full estimation;
\item when \verb"pprior=0" and \verb"ppost=0", the Toolbox analyses the RMSE's for
a multivariate normal MC sample, with covariance matrix based on
the inverse Hessian at the optimum: this analysis is useful when
ML estimation is done (i.e. no Bayesian estimation);
\item when \verb"ppost=1" the Toolbox analyses
the RMSE's for the posterior sample obtained by DYNARE's
Metropolis procedure.
\end{enumerate}

The use of cases 2. and 3. requires an estimation step beforehand!
To facilitate the sensitivity analysis after estimation, the
\verb"dynare_sensitivity" command also allows to indicate some
options of \verb"dynare_estimation". These are:
\begin{itemize}
  \item \verb"datafile"
  \item \verb"mode_file"
  \item \verb"first_obs"
  \item \verb"nobs"
  \item \verb"prefilter"
  \item \verb"presample"
  \item \verb"loglinear"
\end{itemize}


 \vspace{1cm}

\begin{tabular}{r|l|l}
                 option name & default & description  \\ \hline
                \verb"rmse"& 0& 0 = no RMSE analysis\\
                           &  & 1 = do RMSE analysis \\
            \verb"load_rmse"& 0& 0 = make a new RMSE analysis\\
                            &  & 1 = load previous RMSE analysis \\
             \verb"lik_only"& 0& 0 = compute RMSE's for all observed series\\
                            &  & 1 = compute only likelihood and posterior \\
             \verb"var_rmse"& varobs& list of observed series to be considered\\
           \verb"pfilt_rmse"& 0.1& filtering threshold for RMSE's: default it to\\
                            &    & filter the best 10\% for each observed series\\
          \verb"istart_rmse"& 1& start computing RMSE's from \verb"istart_rmse":\\
                            &  & use 2 to avoid big initial error \\
           \verb"alpha_rmse"& 0.002& critical value for Smirnov statistics $d$:\\
                            &   & plot parameters with $d>$\verb"alpha_rmse"\\
          \verb"alpha2_rmse"& 1& critical value for correlation $\rho$\\
                            &  & plot couples of parameters with
                            $|\rho|>$\verb"alpha2_rmse"\\
                 \verb"glue"& 0& prepare for GLUE graphical interface\\\hline
\end{tabular}

\subsection{Screening analysis}
The screening analysis does not require any additional options
with respect to those listed in the `Sampling options':
\verb"morris", \verb"morris_nliv", \verb"morris_ntra". The Toolbox
performs all the analyses required and displays results.


\subsection{Identification analysis (under development)}
Setting the option \verb"identification=1", an identification
analysis based on theoretical moments is performed. Sensitivity plots are provided that
allow to infer which parameters are most likely to be less
identifiable.

Prerequisite for properly running all the identification routines, is the keyword
\verb"identification;" in the DYNARE model file. This keyword triggers the computation of analytic derivatives of the model with respect to estimated parameters and shocks. This is required for options \verb"morris=2" below, which implement \cite{Iskrev2009} identification analysis.
\vspace{1cm}


\begin{tabular}{r|l|l}
                option name & default & description  \\ \hline
       \verb"identification"& 0 & 0 = no identification analysis  \\
                            &   & 1 = performs identification analysis:\\
                            &   & this forces \verb"redform"=0 and default\verb"morris"=1\\
               \verb"morris"& 1 & 1 = Screening analysis (Type II error)\\
                            &   & 0 = ANOVA Mapping (Type I error)\\
                            &   & 2 = Analytic derivatives (similar to Type II error)\\
          \verb"morris_nliv"& 6 & number of levels in Morris design\\
          \verb"morris_ntra"& 20& number of trajectories in Morris design\\
          \verb"trans_ident"& 0 & data transformation for ANOVA Mapping\\
                            &   & 0 = no transformation\\
                            &   & 1 = log-transformation\\
                            &   & 2 = rank-transformation\\\hline
\end{tabular}

\vspace{1cm}
\noindent For example, the following commands in the DYNARE model file

\vspace{1cm}
\noindent\verb"identification;"\\
\verb"dynare_sensitivity(identification=1, morris=2);"

\vspace{1cm}
\noindent trigger the identification analysis using analytic derivatives \citep{Iskrev2009}, jointly with the mapping of the acceptable region.

The identification analysis with derivatives can also be triggered by the commands

\vspace{1cm}
\noindent\verb"identification;"\\
\verb"dynare_identification;"

\vspace{1cm}
This does not do the mapping of acceptable regions for the model and uses the standard random sampler of DYNARE. It completely offsets any use of the sensitivity analysis toolbox.
\newpage
\section{Directory structure}
Sensitivity analysis results are saved on the hard-disk of the
computer. The Toolbox uses a dedicated folder called \verb"GSA",
located in \\
\\
\verb"<DYNARE_file>\GSA", \\
\\
where \verb"<DYNARE_file>.mod" is the name of the DYNARE model
file.

\subsection{Binary data files}
A set of binary data files is saved in the \verb"GSA" folder:
\begin{description}
\item[]\verb"<DYNARE_file>_prior.mat": this file stores
information about the analyses performed sampling from the prior
ranges, i.e. \verb"pprior=1" and \verb"ppost=0";
\item[]\verb"<DYNARE_file>_mc.mat": this file stores
information about the analyses performed sampling from
multivariate normal, i.e. \verb"pprior=0" and \verb"ppost=0";
\item[]\verb"<DYNARE_file>_post.mat": this file stores information
about analyses performed using the Metropolis posterior sample,
i.e. \verb"ppost=1".
\end{description}

\begin{description}
\item[]\verb"<DYNARE_file>_prior_*.mat": these files store
the filtered and smoothed variables for the prior MC sample,
generated when doing  RMSE analysis (\verb"pprior=1" and
\verb"ppost=0");
\item[]\verb"<DYNARE_file>_mc_*.mat": these files store
the filtered and smoothed variables for the multivariate normal MC
sample, generated when doing  RMSE analysis (\verb"pprior=0" and
\verb"ppost=0").
\end{description}

\subsection{Stability analysis}
Figure files \verb"<DYNARE_file>_prior_*.fig" store results for
the stability mapping from prior MC samples:
\begin{description}
\item[]\verb"<DYNARE_file>_prior_stab_SA_*.fig": plots of the Smirnov
test analyses confronting the cdf of the sample fulfilling
Blanchard-Kahn conditions with the cdf of the rest of the sample;
\item[]\verb"<DYNARE_file>_prior_stab_indet_SA_*.fig": plots of the Smirnov
test analyses confronting the cdf of the sample producing
indeterminacy with the cdf of the original prior sample;
\item[]\verb"<DYNARE_file>_prior_stab_unst_SA_*.fig": plots of the Smirnov
test analyses confronting the cdf of the sample producing unstable
(explosive roots) behaviour with the cdf of the original prior
sample;
\item[]\verb"<DYNARE_file>_prior_stable_corr_*.fig": plots of
bivariate projections of the sample fulfilling Blanchard-Kahn
conditions;
\item[]\verb"<DYNARE_file>_prior_indeterm_corr_*.fig": plots of
bivariate projections of the sample producing indeterminacy;
\item[]\verb"<DYNARE_file>_prior_unstable_corr_*.fig":  plots of
bivariate projections of the sample producing instability;
\item[]\verb"<DYNARE_file>_prior_unacceptable_corr_*.fig": plots of
bivariate projections of the sample producing unacceptable
solutions, i.e. either instability or indeterminacy or the
solution could not be found (e.g. the steady state solution could
not be found by the solver).
\end{description}
Similar conventions apply for \verb"<DYNARE_file>_mc_*.fig" files,
obtained when samples from multivariate normal are used.

\subsection{RMSE analysis}
Figure files \verb"<DYNARE_file>_rmse_*.fig" store results for the
RMSE analysis.
\begin{description}
\item[]\verb"<DYNARE_file>_rmse_prior*.fig": save results for
the analysis using prior MC samples;
\item[]\verb"<DYNARE_file>_rmse_mc*.fig": save results for
the analysis using multivariate normal MC samples;
\item[]\verb"<DYNARE_file>_rmse_post*.fig": save results for
the analysis using Metropolis posterior samples.
\end{description}

The following types of figures are saved (we show prior sample to
fix ideas, but the same conventions are used for multivariate
normal and posterior):
\begin{description}
\item[]\verb"<DYNARE_file>_rmse_prior_*.fig": for each parameter, plots the cdf's
corresponding to the best 10\% RMES's of each observed series;
\item[]\verb"<DYNARE_file>_rmse_prior_dens_*.fig": for each parameter, plots the pdf's
corresponding to the best 10\% RMES's of each observed series;
\item[]\verb"<DYNARE_file>_rmse_prior_<name of observedseries>_corr_*.fig": for each observed series plots the
bi-dimensional projections of samples with the best 10\% RMSE's,
when the correlation is significant;
\item[]\verb"<DYNARE_file>_rmse_prior_lnlik*.fig": for each observed
series, plots \emph{in red} the cdf of the log-likelihood
corresponding to the best 10\% RMSE's, \emph{in green} the cdf of
the rest of the sample and \emph{in blue }the cdf of the full
sample; this allows to see the  presence of some idiosyncratic
behaviour;
\item[]\verb"<DYNARE_file>_rmse_prior_lnpost*.fig": for each observed
series, plots \emph{in red} the cdf of the log-posterior
corresponding to the best 10\% RMSE's, \emph{in green} the cdf of
the rest of the sample and \emph{in blue }the cdf of the full
sample; this allows to see idiosyncratic behaviour;
\item[]\verb"<DYNARE_file>_rmse_prior_lnprior*.fig": for each observed
series, plots \emph{in red} the cdf of the log-prior corresponding
to the best 10\% RMSE's, \emph{in green} the cdf of the rest of
the sample and \emph{in blue }the cdf of the full sample; this
allows to see idiosyncratic behaviour;
\item[]\verb"<DYNARE_file>_rmse_prior_lik_SA_*.fig": when
\verb"lik_only=1", this shows the Smirnov tests for the filtering
of the best 10\% log-likelihood values;
\item[]\verb"<DYNARE_file>_rmse_prior_post_SA_*.fig": when
\verb"lik_only=1", this shows the Smirnov test for the filtering
of the best 10\% log-posterior values.
\end{description}

\subsection{Reduced form mapping}
In the case of the mapping of the reduced form solution, synthetic
figures are saved in the \verb"\GSA" folder:

\begin{description}
\item[]\verb"<DYNARE_file>_redform_<endo name>_vs_lags_*.fig":
shows bar charts of the sensitivity indices for the \emph{ten most
important} parameters driving the reduced form coefficients of the
selected endogenous variables (\verb"namendo") versus lagged
endogenous variables (\verb"namlagendo"); suffix \verb"log"
indicates the results for  log-transformed entries;
\item[]\verb"<DYNARE_file>_redform_<endo name>_vs_shocks_*.fig":
shows bar charts of the sensitivity indices for the \emph{ten most
important} parameters driving the reduced form coefficients of the
selected endogenous variables (\verb"namendo") versus exogenous
variables (\verb"namexo"); suffix \verb"log" indicates the results
for  log-transformed entries;
\item[]\verb"<DYNARE_file>_redform_GSA(_log).fig": shows bar chart of
all sensitivity indices for each  parameter: this allows to notice
parameters that have a minor effect for \emph{any} of the reduced
form coefficients,
\end{description}

Detailed results of the analyses are shown in the subfolder
\verb"\GSA\redform_stab", where the detailed results of the
estimation of the single functional relationships between
parameters $\theta$ and reduced form coefficient are stored in
separate directories named as:
\begin{description}
\item[]\verb"<namendo>_vs_<namlagendo>": for the entries of the
transition matrix;
\item[]\verb"<namendo>_vs_<namexo>": for entries of the matrix of
the shocks.
\end{description}
Moreover, analyses for log-transformed entries are denoted with
the following suffixes ($y$ denotes the generic reduced form
coefficient):
\begin{description}
\item[]\verb"log": $y^*=\log(y)$;
\item[]\verb"minuslog": $y^*=\log(-y)$;
\item[]\verb"logsquared": $y^*=\log(y^2)$ for symmetric fat tails;
\item[]\verb"logskew": $y^*=\log(|y+\lambda|)$ for asymmetric fat tails.
\end{description}
The optimal type of transformation is automatically selected
without the need of any user's intervention.

\subsection{Screening analysis}
The results of the screening analysis with Morris sampling design
are stored in the subfolder \verb"\GSA\SCREEN". The data file
\verb"<DYNARE_file>_prior" stores all the information of the
analysis (Morris sample, reduced form coefficients, etc.).

Screening analysis merely concerns reduced form coefficients.
Similar synthetic bar charts as for the reduced form analysis with
MC samples are saved:
\begin{description}
\item[]\verb"<DYNARE_file>_redform_<endo name>_vs_lags_*.fig":
shows bar charts of the elementary effect tests for the \emph{ten
most important} parameters driving the reduced form coefficients
of the selected endogenous variables (\verb"namendo") versus
lagged endogenous variables (\verb"namlagendo");
\item[]\verb"<DYNARE_file>_redform_<endo name>_vs_shocks_*.fig":
shows bar charts of the elementary effect tests for the \emph{ten
most important} parameters driving the reduced form coefficients
of the selected endogenous variables (\verb"namendo") versus
exogenous variables (\verb"namexo");
\item[]\verb"<DYNARE_file>_redform_screen.fig": shows bar chart of
all elementary effect tests for each  parameter: this allows to
identify parameters that have a minor effect for \emph{any} of the
reduced form coefficients.
\end{description}


% ----------------------------------------------------------------
\bibliographystyle{plainnat}
%\bibliographystyle{amsplain}
%\bibliographystyle{alpha}
\bibliography{marco}

\newpage


\end{document}
% ----------------------------------------------------------------
