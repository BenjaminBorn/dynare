\chapter{Estimating DSGE models - basics} \label{ch:estbase}

As in the chapter \ref{ch:solbase}, this chapter is structured around an example. The goal of this chapter is to lead you through the basic functionality in Dynare to estimate models using Bayesian techniques, so that by the end of the chapter you should have the capacity to estimate a model of your own. This chapter is therefore very practically-oriented and abstracts from the underlying computations that Dynare undertakes to estimate a model; that subject is instead covered in some depth in chapter \ref{ch:estbeh}, while more advanced topics of practical appeal are discussed in chapter \ref{ch:estadv}. \\

\section{Introducing an example} \label{sec:modsetup}
The example introduced in this chapter is particularly basic. This is for two reasons. First, we did not want to introduce yet another example in this section; there's enough new material to keep you busy. Instead, we thought it would be easiest to simply continue working with the example introduced in chapter \ref{ch:solbase} with which you are probably already quite familiar. Second, the goal of the example in this chapter is really to explain features in context, but not necessarily to duplicate a ``real life scenario'' you would face when writing a paper. Once you feel comfortable with the content of this chapter, though, you can always move on to chapter \ref{ch:estadv} where you will find a full-fledged replication of a recent academic paper, featuring a non-stationary model. \\

Recall from chapter \ref{ch:solbase} that we are dealing with an RBC model with monopolistic competition. Suppose we had data on business cycle variations of output. Suppose also that we thought our little RBC model did a good job of reproducing reality. We could then use Bayesian methods to estimate the parameters of the model: $\alpha$, the capital share of output, $\beta$, the discount factor, $\delta$, the depreciation rate, $\psi$, the weight on leisure in the utility function, $\rho$, the degree of persistence in productivity, and $\epsilon$, the markup parameter. Note that in Bayesian estimation, the condition for undertaking estimation is that there be at least as many shocks as there are observables (a less stringent condition than for maximum likelihood estimation). It may be that this does not allow you to identify all your parameters - yielding posterior distributions identical to prior distributions - but the Bayesian estimation procedure would still run successfully. Let's see how to go about doing this. 

\section{Declaring variables and parameters}
To input the above model into Dynare for estimation purposes, we must first declare the model's variables in the preamble of the .mod file. This is done exactly as described in chapter \ref{ch:solbase} on solving DSGE models. We thus begin the .mod file with:\\
\\
\texttt{var y c k i l y\_l w r  z;\\
varexo e;\\
\\
parameters beta psi delta alpha rho epsilon;\\}\\
\\

\section{Declaring the model}
Suppose that the equation of motion of technology is a \textbf{stationary} AR(1) with an autoregressive parameter, $\rho$, less than one. The model's variables would therefore be stationary and we can proceed without complications. The alternative scenario with \textbf{non-stationary variables} is more complicated and dealt with in chapter \ref{ch:estadv} (in the additional example). In the stationary case, our model block would look exactly as in chater \ref{ch:solbase}:\\
\\
\texttt{model;\\
  (1/c) = beta*(1/c(+1))*(1+r(+1)-delta);\\
  psi*c/(1-l) = w;\\
  c+i = y;\\
  y = (k(-1)\textasciicircum alpha)*(exp(z)*l)\textasciicircum (1-alpha);\\
  w = y*((epsilon-1)/epsilon)*(1-alpha)/l;\\
  r = y*((epsilon-1)/epsilon)*alpha/k(-1);\\
  i = k-(1-delta)*k(-1);\\
  y\_l = y/l;\\
  z = rho*z(-1)+e;\\
end;}\\
\\

\section{Declaring observable variables}
This should not come as a surprise. Dynare must know which variables are observable for the estimation procedure. \textsf{\textbf{NOTE!}} These variables must be available in the data file, as explained in section \ref{sec:estimate} below. For the moment, we write:\\
\\
\texttt{varobs y;}\\

\section{Specifying the steady state} \label{sec:ssest}
Before Dynare estimates a model, it first linearizes it around a steady state. Thus, a steady state must exist for the model and although Dynare can calculate it, we must give it a hand by declaring approximate values for the steady state. This is just as explained in details and according to the same syntax outlined in chapter \ref{ch:solbase}, covering the \texttt{initval}, \texttt{steady} and \texttt{check} commands. In fact, as this chapter uses the same model as that outlined in chapter \ref{ch:solbase}, the steady state block will look exactly the same.\\

\textsf{\textbf{TIP!}} During estimation, in finding the posterior mode, Dynare recalculates the steady state of the model at each iteration of the optimization routine (more on this later), based on the newest round of parameters available. Thus, by providing approximate initial values and relying solely on the built-in Dynare algorithm to find the steady state (a numerical procedure), you will significantly slow down the computation of the posterior mode. Dynare will end up spending 60 to 70\% of the time recalculating steady states. It is much more efficient to write an external \textbf{Matlab steady state file} and let Dynare use that file to find the steady state of your model by algebraic procedure. For more details on writing an external Matlab file to find your model's steady state, please refer to section \ref{sec:findsteady} of chapter \ref{ch:solbase}. \\


\section{Declaring priors}
Priors play an important role in Bayesian estimation and consequently deserve a central role in the specification of the .mod file. Priors, in Bayesian estimation, are declared as a distribution. The general syntax to introduce priors in Dynare is the following:\\
\\
\texttt{estimated\_params;\\
PARAMETER NAME, PRIOR\_SHAPE, PRIOR\_MEAN, PRIOR\_STANDARD\_ERROR [, PRIOR 3$^{\textrm{rd}}$ PARAMETER] [,PRIOR 4$^{\textrm{th}}$ PARAMETER] ; \\
end; }\\
\\
where the following table defines each term more clearly\\
\\
    \begin{tabular}{llc}
    \textbf{\underline{PRIOR\_SHAPE}} & \textbf{\underline{Corresponding distribution}} & \textbf{\underline{Range}} \\
      NORMAL\_PDF & $N(\mu,\sigma)$ & $\mathbb{R}$\\
      GAMMA\_PDF &  $G_2(\mu,\sigma,p_3)$ & $[p_3,+\infty)$\\
        BETA\_PDF & $B(\mu,\sigma,p_3,p_4)$ & $[p_3,p_4]$\\
        INV\_GAMMA\_PDF & $IG_1(\mu,\sigma)$ & $\mathbb R^+$\\
UNIFORM\_PDF& $U(p_3,p_4)$ &  $[p_3,p_4]$
    \end{tabular}\\
    \\
    \\
    where $\mu$ is the \texttt{PRIOR\_MEAN}, $\sigma$ is the \texttt{PRIOR\_STANDARD\_ERROR}, $p_3$ is the \texttt{PRIOR 3$^{\textrm{rd}}$ PARAMETER} (whose default is 0) and $p_4$ is the \texttt{PRIOR 4$^{\textrm{th}}$ PARAMETER} (whose default is 1).  \textsf{\textbf{TIP!}} When specifying a uniform distribution between 0 and 1 as a prior for a parameter, say $\alpha$, you therefore have to put two empty spaces for parameters $\mu$ and $\sigma$, and then specify parameters $p3$ and $p4$, since the uniform distribution only takes $p3$ and $p4$ as arguments. For instance, you would write \texttt{alpha, uniform\_pdf, , , 0,1;} \\
    
For a more complete review of all possible options for declaring priors, as well as the syntax to declare priors for maximum likelihood estimation (not Bayesian), see the \href{http://www.dynare.org/documentation-and-support/manual}{Reference Manual}. Note also that if some parameters in a model are calibrated and are not to be estimated, you should declare them as such, by using the \texttt{parameters} command and its related syntax, as explained in chapter \ref{ch:solbase}. \\

\textsf{\textbf{TIP!}} Choosing the appropriate prior for your parameters is a tricky, yet very important endeavor. It is worth spending time on your choice of priors and to test the robustness of your results to your priors. Some considerations may prove helpful. First, think about the domain of your prior over each parameter. Should it be bounded? Should it be opened on either or both sides? Remember also that if you specify a probability of zero over a certain domain in your prior, you will necessarily also find a probability of zero in your posterior distribution. Then, think about the shape of your prior distribution. Should it be symmetric? Skewed? If so, on which side? You may also go one step further and build a distribution for each of your parameters in your mind. Ask yourself, for instance, what is the probability that your parameter is bigger than a certain value, and repeat the exercise by incrementally decreasing that value. You can then pick the standard distribution that best fits your perceived distribution. Finally, instead of describing here the shapes and properties of each standard distribution available in Dynare, you are instead encouraged to visualize these distributions yourself, either in a statistics book or on the Web. \\
    
\textsf{\textbf{TIP!}} It may at times be desirable to estimate a transformation of a parameter appearing in the model, rather than the parameter itself. In such a case, it is possible to declare the parameter to be estimated in the parameters statement and to define the transformation at the top of the model section, as a Matlab expression, by adding a pound sign (\#) at the beginning of the corresponding line. For example, you may find it easier to define a prior over the discount factor, $\beta$, than its inverse which often shows up in Euler equations. Thus you would write:\\
\\
\texttt{model;\\
      \# sig = 1/bet;\\
      c = sig*c(+1)*mpk;\\
      end;\\
\\
      estimated\_params;\\
      bet,normal\_pdf,1,0.05;\\
      end;}\\

\textsf{\textbf{TIP!}} Finally, another useful command to use is the \texttt{estimated\_params\_init} command which declares numerical initial values for the optimizer when these are different from the prior mean. This is especially useful when redoing an estimation - if the optimizer got stuck the first time around, or needing a greater number of draws in the Metropolis-Hastings algorithm - and wanting to enter the posterior mode as initial values for the parameters instead of a prior. The \href{http://www.dynare.org/documentation-and-support/manual}{Reference Manual} gives more details as to the exact syntax of this command.\\

Coming back to our basic example, we would write:\\
\\
\texttt{estimated\_params;\\
alpha, beta\_pdf, 0.35, 0.02; \\ 
beta, beta\_pdf, 0.99, 0.002; \\
delta, beta\_pdf, 0.025, 0.003;\\
psi, gamma\_pdf, 1.75, 0.02;\\
rho, beta\_pdf, 0.95, 0.05;\\
epsilon, gamma\_pdf, 10, 0.003;\\
stderr e, inv\_gamma\_pdf, 0.01, inf;\\
end;}\\


\section{Launching the estimation} \label{sec:estimate}
To ask Dynare to estimate a model, all that is necessary is to add the command \texttt{estimation} at the end of the .mod file. Easy enough. But the real complexity comes from the options available for the command (to be entered in parentheses and sequentially, separated by commas, after the command \texttt{estimation}). Below, we list the most common and useful options, and encourage you to view the \href{http://www.dynare.org/documentation-and-support/manual}{Reference Manual} for a complete list. \\
\begin{enumerate}
\item datafile = FILENAME: the datafile (a .m file, a .mat file, or an .xls file). Note that observations do not need to show up in any order, but vectors of observations need to be named with the same names as those in \texttt{var\_obs}. In Excel files, for instance, observations could be ordered in columns, and variable names would show up in the first cell of each column.
\item nobs = INTEGER: the number of observations to be used (default: all observations in the file) 
\item first\_obs = INTEGER: the number of the first observation to be used (default = 1). This is useful when running loops, or instance, to divide the observations into sub-periods. 
\item prefilter = 1: the estimation procedure demeans the data (default=0, no prefiltering). This is useful if model variables are in deviations from steady state, for instance, and therefore have zero mean. Demeaning the observations would also impose a zero mean on the observed variables.  
\item nograph: no graphs should be plotted.
\item conf\_sig = \{INTEGER | DOUBLE \}: the level for the confidence intervals reported in the results (default = 0.90) 
\item mh\_replic = INTEGER: number of replication for Metropolis Hasting algorithm. For the time being, mh\_replic 
should be larger than 1200 (default = 20000) 
\item mh\_nblocks = INTEGER: number of parallel chains for Metropolis Hasting algorithm (default = 2). Despite this low default value, it is advisable to work with a higher value, such as 5 or more. This improves the computation of between group variance of the parameter means, one of the key criteria to evaluate the efficiency of the Metropolis-Hastings to evaluate the posterior distribution. More details on this subject appear in Chapter \ref{ch:estadv}.  
\item mh\_drop = DOUBLE: the fraction of initially generated parameter vectors to be dropped before using posterior 
simulations (default = 0.5; this means that the first half of the draws from the Metropolis-Hastings are discarded). 
\item mh\_jscale = DOUBLE: the scale to be used for the jumping distribution in MH algorithm. The default value is 
rarely satisfactory. This option must be tuned to obtain, ideally, an acceptance rate of 25\% in the Metropolis- Hastings algorithm (default = 0.2). The idea is not to reject or accept too often a candidate parameter; the literature has settled on a value of between 0.2 and 0.4. If the acceptance rate were too high, your Metropolis-Hastings iterations would never visit the tails of a distribution, while if it were too low, the iterations would get stuck in a subspace of the parameter range. Note that the acceptance rate drops if you increase the scale used in the jumping distribution and vice a versa.
\item mh\_init\_scale=DOUBLE: the scale to be used for drawing the initial value of the Metropolis-Hastings chain (default=2*mh\_jscale). The idea here is to draw initial values from a stretched out distribution in order to maximize the chances of these values not being too close together, which would defeat the purpose of running several blocks of Metropolis-Hastings chains.
\item mode\_file=FILENAME: name of the file containing previous value for the mode. When computing the mode, 
Dynare stores the mode (xparam1) and the hessian (hh) in a file called MODEL NAME\_mode. This is a particularly helpful option to speed up the estimation process if you have already undertaken initial estimations and have values of the posterior mode. 
\item mode\_compute=INTEGER: specifies the optimizer for the mode computation. 
\subitem 0: the mode isn�t computed. mode\_file must be specified 
\subitem 1: uses Matlab fmincon (see the \href{http://www.dynare.org/documentation-and-support/manual}{Reference Manual} to set options for this command).
\subitem 2: uses Lester Ingber�s Adaptive Simulated Annealing. 
\subitem 3: uses Matlab fminunc. 
\subitem 4 (default): uses Chris Sim�s csminwel. 
\item mode\_check: when mode\_check is set, Dynare plots the minus of the posterior density for values around the computed mode 
for each estimated parameter in turn. This is helpful to diagnose problems with the optimizer. A clear indication of a problem would be that the mode is not at the trough (bottom of the minus) of the posterior distribution. 
\item load\_mh\_file: when load\_mh\_file is declared, Dynare adds to previous Metropolis-Hastings simulations instead 
of starting from scratch. Again, this is a useful option to speed up the process of estimation.
\item nodiagnostic: doesn�t compute the convergence diagnostics for Metropolis-Hastings (default: diagnostics are computed and 
displayed). Actually seeing if the various blocks of Metropolis-Hastings runs converge is a powerful and useful option to build confidence in your model estimation. More details on these diagnostics are given in Chapter \ref{ch:estadv}.
\item bayesian\_irf: triggers the computation of the posterior distribution of impulse response functions (IRFs). The length of the IRFs are controlled by the irf option, as specified in chapter \ref{ch:solbase} when discussing the options for \texttt{stoch\_simul}. To build the posterior distribution of the IRFs, Dynare pulls parameter and shock values from the corresponding estimated distributions and, for each set of draws, generates an IRF. Repeating this process often enough generates a distribution of IRFs. \textsf{\textbf{TIP!}} If you stop the estimation procedure after calculating the posterior mode, or carry out maximum likelihood estimation, only the corresponding parameter estimates will be used to generate the IRFs. If you instead carry out a full Metropolis-Hastings estimation, on the other hand, the IRFs will use the parameters the posterior distributions, including the variance of the shocks. 
\item All options available for stoch\_simul can simply be added to the above options, separated by commas. To view a list of these options, either see the \href{http://www.dynare.org/documentation-and-support/manual}{Reference Manual} or section \ref{sec:compute} of chapter \ref{ch:solbase}.
\item moments\_varendo: triggers the computation of the posterior distribution of the theoretical moments of the endogenous variables as in \texttt{stoch\_simul} (the posterior distribution of the variance decomposition is also included). ** will be implemented shortly - if not already - in Dynare version 4. 
\item filtered\_vars: triggers the computation of the posterior distribution of filtered endogenous variables and shocks. See the note below on the difference between filtered and smoothed shocks. ** will be implemented shortly - if not already - in Dynare version 4. 
\item smoother: triggers the computation of the posterior distribution of smoothed endogenous variables and shocks. Smoothed shocks are a reconstruction of the values of unobserved shocks over the sample, using all the information contained in the sample of observations. Filtered shocks, instead, are built only based on knowing past information. To calculate one period ahead prediction errors, for instance, you should use filtered, not smoothed variables.
\item forecast = INTEGER: computes the posterior distribution of a forecast on INTEGER periods after the end of the sample used in estimation. The corresponding graph includes one confidence interval describing uncertainty due to parameters and one confidence interval describing uncertainty due to parameters and future shocks. Note that Dynare cannot forecast out of the posterior mode. You need to run Metropolis-Hastings iterations before being able to run forecasts on an estimated model. Finally, running a forecast is very similar to an IRF, as in \texttt{bayesian\_irf}, except that the forecast does not begin at a steady state, but simply at the point corresponding to the last set of observations. The goal of undertaking a forecast is to see how the system returns to steady state from this starting point. Of course, as observation do not exist for all variables, those necessary are reconstructed by sampling out of the posterior distribution of parameters. Again, repeating this step often enough yields a posterior distribution of the forecast. ** will be implemented shortly - if not already - in Dynare version 4. 
\end{enumerate}

\textsf{\textbf{TIP!}} Before launching estimation it is a good idea to make sure that your model is correctly declared, that a steady state exists and that it can be simulated for at least one set of parameter values. You may therefore want to create a test version of your .mod file. In this test file, you would comment out or erase the commands related to estimation, remove the prior estimates for parameter values and replace them with actual parameter values in the preamble, remove any non-stationary variables from your model, add a \texttt{shocks} block, make sure you have \texttt{steady} and possibly \texttt{check} following the \texttt{initval} block if you do not have exact steady state values and run a simulation using \texttt{stoch\_simul} at the end of your .mod file. Details on model solution and simulation can be found in Chapter \ref{ch:solbase}. \\

Finally, coming back to our example, we could choose a standard option:\\
\\
\texttt{estimation(datafile=simuldataRBC,nobs=200,first\_obs=500,\\
mh\_replic=2000,mh\_nblocks=2,mh\_drop=0.45,mh\_jscale=0.8,\\
mode\_compute=4); }\\

This ends our description of the .mod file.

\section{The complete .mod file}
To summarize and to get a complete perspective on our work so far, here is the complete .mod file for the estimation of our very basic model. You can find the corresponding file in the \textsl{models} folder under \textsl{UserGuide} in your installation of Dynare. The file is called \textsl{RBC\_Est.mod}.\\
\\
\texttt{var y c k i l y\_l w r  z;\\
varexo e;\\
parameters beta psi delta alpha rho epsilon;\\
\\
model;\\
  (1/c) = beta*(1/c(+1))*(1+r(+1)-delta);\\
  psi*c/(1-l) = w;\\
  c+i = y;\\
  y = (k(-1)\textasciicircum alpha)*(exp(z)*l)\textasciicircum (1-alpha);\\
  w = y*((epsilon-1)/epsilon)*(1-alpha)/l;\\
  r = y*((epsilon-1)/epsilon)*alpha/k(-1);\\
  i = k-(1-delta)*k(-1);\\
  y\_l = y/l;\\
  z = rho*z(-1)+e;\\
end;\\
\\
varobs y;\\
\\
initval;\\
  k = 9;\\
  c = 0.7;\\
  l = 0.3;\\
  w = 2.0;\\
  r = 0;\\
  z = 0; \\
  e = 0;\\
end;\\
\\
steady;\\
check;\\
\\
estimated\_params;\\
alpha, beta\_pdf, 0.35, 0.02; \\ 
beta, beta\_pdf, 0.99, 0.002; \\
delta, beta\_pdf, 0.025, 0.003;\\
psi, gamma\_pdf, 1.75, 0.02;\\
rho, beta\_pdf, 0.95, 0.05;\\
epsilon, gamma\_pdf, 10, 0.003;\\
stderr e, inv\_gamma\_pdf, 0.01, inf;\\
end;\\
\\
estimation(datafile=simuldataRBC,nobs=200,first\_obs=500,\\
mh\_replic=2000,mh\_nblocks=2,mh\_drop=0.45,mh\_jscale=0.8,\\
mode\_compute=4); }
\\


\section{Interpreting output}

As in the case of model solution and simulation, Dynare returns both tabular and graphical output. On the basis of the options entered in the example .mod file above, Dynare will display the following results.\\

\subsection{Tabular results}
The first results to be displayed (and calculated from a chronological standpoint) are the steady state results. Note the dummy values of 1 for the non-stationary variables Y\_obs and P\_obs. These results are followed by the eigenvalues of the system, presented in the order in which the endogenous variables are declared at the beginning of the .mod file. The table of eigenvalues is completed with a statement about the Blanchard-Kahn condition being met - hopefully!\\

The next set of results are for the numerical iterations necessary to find the posterior mode, as explained in more details in Chapter \ref{ch:estadv}. The improvement from one iteration to the next reaches zero, Dynare give the value of the objective function (the posterior Kernel) at the mode and displays two important table summarizing results from posterior maximization.\\

The first table summarizes results for parameter values. It includes: prior means, posterior mode, standard deviation and t-stat of the mode (based on the assumption of a Normal, probably erroneous when undertaking Bayesian estimation, as opposed to standard maximum likelihood), as well as the prior distribution and standard deviation (pstdev). It is followed by a second table summarizing the same results for the shocks. It may be entirely possible that you get an infinite value for a standard deviation, this is simply the limit case of the inverse gamma distribution.\\

\subsection{Graphical results}
** corresponding graphs will be reproduced below.\\

The first figure comes up soon after launching Dynare as little computation is necessary to generate it. The figure returns a graphical representation of the priors for each parameter of interest. \\

The second set of figures displays ``MCMC univariate diagnostics'', where MCMC stands for Monte Carlo Markov Chains. This is the main source of feedback to gain confidence, or spot a problem, with results. Recall that Dynare completes several runs of Metropolis-Hastings simulations (as many as determined in the option \texttt{mh\_nblocks}, each time starting from a different initial value). If the results from one chain are sensible, and the optimizer did not get stuck in an odd area of the parameter subspace, two things should happen. First, results within any of the however many iterations of Metropolis-Hastings simulation should be similar. And second, results between the various chains should be close. This is the idea of what the MCMC diagnostics track. \\

More specifically, the red and blue lines on the charts represent specific measures of the parameter vectors both within and between chains. For the results to be sensible, these should be relatively constant (although there will always be some variation) and they should converge. Dynare reports three measures: ``interval'', being constructed from an 80\% confidence interval around the parameter mean, ``m2'', being a measure of the variance and ``m3'' based on third moments. In each case, Dynare reports both the within and the between chains measures. The figure entitled ``multivariate diagnostic'' presents results of the same nature, except that they reflect an aggregate measure based on the eigenvalues of the variance-covariance matrix of each parameter.\\

In our example above, you can tell that indeed, we obtain convergence and relative stability in all measures of the parameter moments. Note that the horizontal axis represents the number of Metropolis-Hastings iterations that have been undertaken, and the vertical axis the measure of the parameter moments, with the first, corresponding to the measure at the initial value of the Metropolis-Hastings iterations.\\

\textsf{\textbf{TIP!}} If the plotted moments are highly unstable or do not converge, you may have a problem of poor priors. It is advisable to redo the estimation with different priors. If you have trouble coming up with a new prior, try starting with a uniform and relatively wide prior and see where the data leads the posterior distribution. Another approach is to undertake a greater number of Metropolis-Hastings simulations.\\

The first to last figure - figure 6 in our example - displays the most interesting set of results, towards which most of the computations undertaken by Dynare are directed: the posterior distribution. In fact, the figure compares the posterior to the prior distribution (black vs. grey lines). In addition, on the posterior distribution, Dynare plots a green line which represents the posterior mode. These allow you to make statements about your data other than simply concerning the mean and variance of the parameters; you can also discuss the probability that your parameter is larger or smaller than a certain value.\\

\textsf{\textbf{TIP!}} These graphs are of course especially relevant and present key results, but they can also serve as tools to detect problems or build additional confidence in your results. First, the prior and the posterior distributions should not be excessively different. Second, the posterior distributions should be close to normal, or at least not display a shape that is clearly non-normal. Third, the green mode (calculated from the numerical optimization of the posterior kernel) should not be too far away from the mode of the posterior distribution. If not, it is advisable to undertake a greater number of Metropolis-Hastings simulations. \\

The last figure returns the smoothed estimated shocks in a useful illustration to eye-ball the plausibility of the size and frequency of the shocks. The horizontal axis, in this case, represents the number of periods in the sample. One thing to check is the fact that shocks should be centered around zero. That is indeed the case for our example. \\


\begin{comment}
\section{Dealing with non-stationary models}
Many DSGE models will embody a source of non-stationarity. This adds a layer of complication to estimating these models. In Dynare, this involves stationarizing variables, as well as several new lines of instruction. Again, the example below will be as basic as possible, building on the example used above. But you can go through a more complex example in the advanced chapter on DSGE estimation (chapter \ref{ch:estadv}). \\

Below, we will introduce the non-stationarity with a stochastic trend in technology diffusion. But in order to linearize and solve the DSGE model - a necessary first step before estimating it - we must have a steady state. Thus, we must write the model using stationary variables. Our task will then be to link these stationary variables to the data which, according to the model, would be non-stationary. \textsf{\textbf{NOTE!}} But whatever the data, in Dynare, \textbf{the model must be made stationary}. \\

\subsection{The source of non-stationarity}



\section{Specifying observable variable characteristics}
A natural extension of our discussion above is to declare, in Dynare, which variables are observable and which are non-stationary. This can be done just after the model block. 

\subsection{Declaring observable variables}
We begin by declaring which of our model's variables are observables. In our .mod file we write\\
\\
\texttt{varobs P\_obs Y\_obs;}\\
\\
to specify that our observable variables are indeed P\_obs and Y\_obs as noted in the section above. \textsf{\textbf{NOTE!}} The number of unobserved variables (number of \texttt{var} minus number of \texttt{varobs}) must be smaller or equal to the number of shocks such that the model be estimated. If this is not the case, you should add measurement shocks to your model where you deem most appropriate. \\

\subsection{Declaring trends in observable variables}

Recall that we decided to work with the non-stationary observable variables in levels. Both output and prices exhibit stochastic trends. This can be seen explicitly by taking the difference of logs of output and prices to compute growth rates. In the case of output, we make use of the usual (by now!) relationship $Y_t=\widehat Y_t \cdot A_t$. Taking logs of both sides and subtracting the same equation scrolled back one period, we find:
\[
\Delta \ln Y_t = \Delta \ln \widehat Y_t + \gamma + \epsilon_{A,t}
\]
emphasizing clearly the drift term $\gamma$, whereas we know $\Delta \ln \widehat Y_t$ is stationary in steady state. \\

In the case of prices, we apply the same manipulations to show that:
\[
\Delta \ln P_t = \Delta \ln \widehat P_t + \ln m_{t-1} - \gamma - \epsilon_{A,t}
\]

Note from the original equation of motion of $\ln m_t$ that in steady state, $\ln m_t=\ln m^*$, so that the drift terms in the above equation are $\ln m^* - \gamma$.\footnote{This can also be see from substituting for $\ln m_{t-1}$ in the above equation with the equation of motion of $\ln m_t$ to yield: $\Delta \ln P_t = \Delta \ln \widehat P_t + \ln m^* + \rho(\ln m_{t-2}-\ln m^*) + \epsilon_{M,t} - \gamma - \epsilon_{A,t}$ where all terms on the right hand side are constant, except for $\ln m^*$ and $\gamma$.} \\

In Dynare, any trends, whether deterministic or stochastic (the drift term) must be declared up front. In the case of our example, we therefore write (in a somewhat cumbersome manner)\\
\\
\texttt{observation\_trends;\\
P\_obs (log(exp(gam)/mst));\\
Y\_obs (gam);\\
end;}\\

In general, the command \texttt{observation\_trends} specifies linear trends as a function of model parameters for the observed variables in the model.\\

\subsection{Declaring unit roots, if any, in observable variables}

And finally, since P\_obs and Y\_obs inherit the unit root characteristics of their driving variables, technology and money, we must tell Dynare to use a diffuse prior (infinite variance) for their initialization in the Kalman filter. Note that for stationary variables, the unconditional covariance matrix of these variables is used for initialization. The algorithm to compute a true diffuse prior is taken from \citet{DurbinKoopman2001}. To give these instructions to Dynare, we write in the .mod\\
\\
\texttt{unit\_root\_vars = \{'P\_obs'; 'Y\_obs' \};}\\
\\
\textsf{\textbf{NOTE!}} You don't need to declare unit roots for any non-stationary model. Unit roots are only related to stochastic trends. You don't need to use a diffuse initial condition in the case of a deterministic trend, since the variance is finite.\\

\section{SS}

UNIT ROOT SS: The only difference with regards to model estimation is the non-stationary nature of our observable variables. In this case, the declaration of the steady state must be slightly amended in Dynare. (** fill in with the new functionality from Dynare version 4, currently the answer is to write a separate Matlab file which solves the steady state for the stationary variables and returns dummies for the non stationary variables). 



 \end{comment}

 






