\chapter{Installing Dynare} \label{ch:inst}

\section{Dynare versions}

Dynare runs on both \href{http://www.mathworks.com/products/matlab/}{\textbf{MATLAB}} and \href{http://www.octave.org}{\textbf{GNU Octave}}.

There used to be versions of Dynare for \textbf{Scilab} and \textbf{Gauss}. Development of the Scilab version stopped after Dynare version 3.02 and that for Gauss after Dynare version 1.2.

This User Guide will exclusively \textbf{focus on Dynare version 4.0 and later}.

You may also be interested by another program, \textbf{Dynare++}, which is a standalone C++ program specialized in computing k-order approximations of dynamic stochastic general equilibrium models. Note that Dynare++ is distributed along with Dynare since version 4.1. See the \href{http://www.dynare.org/documentation-and-support/dynarepp}{Dynare++ webpage} for more information. 

\section{System requirements}
Dynare can run on Microsoft Windows, as well as Unix-like operating systems, in particular GNU/Linux and Mac OS X. If you have questions about the support of a particular platform, please ask your question on \href{http://www.dynare.org/phpBB3}{\textbf{Dynare forums}}.

To run Dynare, it is recommended to allocate at least 256MB of RAM to the platform running Dynare, although 512MB is preferred. Depending on the type of computations required, like the very processor intensive Metropolis Hastings algorithm, you may need up to 1GB of RAM to obtain acceptable computational times.

\section{Installing Dynare}

Please refer to the section entitled ``Installation and configuration'' in the \href{http://www.dynare.org/documentation-and-support/manual}{Dynare reference manual}.

\section{MATLAB particularities}

A question often comes up: what special MATLAB toolboxes are necessary to run Dynare? In fact, no additional toolbox is necessary for running most of Dynare, except maybe for optimal simple rules (see chapter \ref{ch:ramsey}), but even then remedies exist (see the \href{http://www.dynare.org/phpBB3}{Dynare forums} for discussions on this, or to ask your particular question). But if you do have the `optimization toolbox' installed, you will have additional options for solving for the steady state (solve\_algo option) and for searching for the posterior mode (mode\_compute option), both of which are defined later. 
